\documentclass[letterpaper,12pt]{article}

\usepackage{csm-thesis}
\usepackage{array}
\usepackage{lscape}
\usepackage{longtable}
\usepackage{rotating}
\usepackage[authoryear]{natbib}
\usepackage{hyperref}
\usepackage{float}
\usepackage{multirow}
\usepackage{graphicx,subfig}
\usepackage{listings}
\usepackage{pdflscape} % use ``lscape'' if you are not creating a PDF output
\usepackage{amsmath}
	\title{Joint inversion of multicomponent seismic 3D data and rock physics modeling in Delhi Field}
\degreetitle{Master of Science}
\discipline{Geophysics}
\department{Geophysics}

\author{Carla C. Carvajal Meneses}
\advisor{Dr. Thomas L. Davis}
% Comment out the following line if you do not have a co-advisor:
%\coadvisor{Dr. Secondary B. Advisor}
\dpthead{Dr. Terence K. Young}{Professor and Head}

\begin{document}

\frontmatter

\maketitle
\newpage

\makecopyright{\the\year}
\newpage

%%% Parts of a Thesis - Front Matter - Signature Page (required)

\makesubmittal
\newpage

%%% Parts of a Thesis - Front Matter - Abstract (required)

\begin{abstract}
In order to accomplish the objectives I am proposing a workflow to analyze
  time lapse data in clastic reservoirs by incorporating rock physics modeling
 and multicomponent seismic data.
   
The rock physics static modeling is based on a hybrid  approach that uses theoretic,
 heuristic and empirical models for cemented sandstones (Avseth et al., 2010). 
This model will be validated with the rock laboratory measurements of Vp and Vs for
 Tuscaloosa Formation.  Then I proposed to extend the modeling into dynamic reservoir 
 characterization framework by using fluid and pressure substitution.

4D joint seismic inversion between PP and PS radial data will provide acoustic, shear 
and Vp/Vs ratio information for the first and second monitor more accurately because it 
should add stability to an ill- conditioned problem, that is very sensitive to noise and non-unique.
The initial model of the inversion will be improve by implementing  Dynamic Image Warping
 (DIW) in order to match properly the PP and PS radial horizons.


\end{abstract}

\newpage

%%% Parts of a Thesis - Front Matter - Table of Contents (required)

\tableofcontents

\newpage

\listoffiguresandtables

\newpage

\listofsymbols

\newpage

\addsymbol{Critical porosity}{$\phi_{C}$}
\addsymbol{Bulk modulus of the dry mineral frame}{$K_{s}$}
\addsymbol{Bulk modulus of the cement}{$K_{c}$}
\addsymbol{Shear modulus of the dry mineral frame}{$G_{s}$}
\addsymbol{Shear modulus of the cement}{$G_{c}$}
\addsymbol{Compressional shear modulus of the cement}{$M_{c}$}
\addsymbol{Poisson ratio of the dry mineral frame}{$\upsilon_{s}$}
\addsymbol{Poisson ratio of the cemeny}{$\upsilon_{c}$}
\addsymbol{Clay content}{C}
\addsymbol{Coordination number}{n}




%\addsymbol{absorption cross section}{$\alpha_{\sigma}$}
%\addsymbol{average radius of cylindrical shell}{$c$}
%\addsymbol{activation energy of oxidation reaction of a-C in excited state}{$E^{\ast}_{act}$}

%\listofabbreviations
%\newpage

% Note that you may define abbreviations anywhere in the document, when you re-run LaTeX they
% will be added to the list (just like all other lists)
%\addabbreviation{Bio Force Gun, Model 9000}{BFG9000}
%\addabbreviation{Mammoth Armed Reclamation Vehicle}{MARV}
%\addabbreviation{Stone of Jordan}{SoJ}
%\addabbreviation{Field flow fractionation-inductively coupled plasma-mass\newline spectrometry% with extra% magic and stuff and things
%\addabbreviation{Field flow fractionation-inductively coupled plasma-mass spectrometry% with extra% magic and stuff and things
%}{FFF-ICP-MS}

%%% Parts of a Thesis - Front Matter - Acknowledgments (optional)

\begin{acknowledgments}
To Toto
\end{acknowledgments}
\newpage

%%% Parts of a Thesis - Front Matter - Dedication (optional)

\begin{dedication}
To my love, friend and favourite geophysicist, Toto
\end{dedication}
\newpage

%% Parts of a Thesis - Body

\bodymatter

%\chapter{test bib}

%\cite{ref:A,ref:B,ref:C} o los puedes citar asi~\cite{ref:A} and 

%%%%%%%%%%%%%%%%%%%%%%%%%%%%%%%%%%%%%%%%%%%%%%%%%%%%%%%%%%%%%%%%%%%%%%%%%%%%%%%%%%%%%%%%%%%%%
% If you would like to work on each chapter of your thesis in a separate document then use: %
%%%%%%%%%%%%%%%%%%%%%%%%%%%%%%%%%%%%%%%%%%%%%%%%%%%%%%%%%%%%%%%%%%%%%%%%%%%%%%%%%%%%%%%%%%%%%
\chapter{Introduction}
Delhi Field is a case of study of the Reservoir Characterization Project (RCP)
since 2009, it is an excellent example for an enhanced oil recovery (EOR) 
active monitoring field with 4D multicomponent seismic technologies. This research
project is associated with the RCP Phase XIV, which consists in a multidisciplinary study 
between geology, geophysics and petroleum engineering in order to enhance the 
static and dynamic reservoir characterization to maximize the recovery of the field. 
The operator company, Denbury Resources inc. started the CO$_2$ flooding in 2009 and estimate
recover about 60 million barrels that represents 17\% of the original oil in place
(OOIP) of this field that have already produced 54\% after primary and secondary depletion.


\subsection{Delhi Field}

Delhi Field is located in northeastern Louisiana, 40 miles west of 
Loussiana-Mississippi border \citep{ref:nick} and 35 miles from Monroe (\ref{fig:location}).
Geologically, lies at the western margin of the Mississippi Interior Salt basin
\citep{ref:alam} and at the southern flank of the Monroe uplift \citep{ref:bloomer} 
(\ref{fig:location}). The field is 15 miles long by 2--2.5 miles wide covering an
area of 6200 acres. The OOIP is about 357 MMbbl \citep{ref:powell} and has an API
 gravity of 43.9 (Denbury Resources inc.). 
The field was discovered in 1944, primary and second recovery was about 14\% and 40\%
respectively \citep{ref:bloomer}. The peak production of the field was 18000 barrels per day (BOPD). 
The CO$_2$ continuous flooding program started in November 2009 with the expectation
 of recover an additional 17\%. The first production associated with the injection began
 in march 2010 and the maximum production since the tertiary 
recovery has reach 4,458 barrels BOPD, see \ref{fig:production_new}. 

\csmlongfigure[h!]{location}{./chapters/Chapter1_intro/Figures/location}{5in}{Delhi field
                             is located at northeastern Lousiana. It lies in the southern
                             flank of the Monroe uplift and Mississipi Interior Salt Basin (MISB).
                             Jackson dome is the CO$_2$ source for Delhi field. 
                             (modified from Denbury Resources inc)}

\csmlongfigure[p]{production_new}{./chapters/Chapter1_intro/Figures/production_new}{5.28in}
                  {Production history of Delhi Field. The peak production was reach in
                   1973 with 18,000 BOPD. Note how the production stop in 2005 and it 
                   reactivated after terciary recovery.}


The main reservoir is the Holt Bryant zone that lies at depths between 3,000 and 3,500
feet,  which is composed by Tuscaloosa and Paluxy sandstones, Upper and Lower Cretaceous 
respectively (\ref{fig:column}). The Paluxy represents a progradational deltaic depositional environment
 \citep{ref:nick} with average porosity 30\%, permeability of 1000 mD and it range in thickness
from 30 to 75 feet. The Tuscaloosa can be divided into lower and upper units. The lower unit 
consist on transgressive marine depositional processes \citep{ref:doug}, laterelly discontinuous.
The upper Tuscaloosa represent fluvial depositional environment with the presence of some marine
processes \citep{ref:doug}. The average porosity and permeability of the Tuscaloosa units is 29\%
 and 2700 mD and it range in thickness from 35 to 85 feet. 

The trap of the reservoir is an unconformable stratigraphic trap. The Monroe Gas Rock, 
a hard chalk or  limestone, forms the top seal \citep{ref:powell} along with the Clayton Chalk (\ref{fig:stratigraphy}) 
\citep{ref:nick}.The Jurassic, Smackover Limestone is the most probable source rock \citep{ref:mancini}.

For a better understanding in geology of the field the reader is referred 
to \citep{ref:nick} and \citep{ref:doug}.

\csmlongfigure[h!]{column}{./chapters/Chapter1_intro/Figures/column.pdf}{5in}{Stratigraphic 
                           column in Delhi Field, modified from \cite{ref:nick}}

\csmlongfigure[h!]{stratigraphy}{./chapters/Chapter1_intro/Figures/stratigraphy}{5.78in}{ Northwest to Southeast
                                 Seismic cross-section showing the the trap of the reservoir. Highlighted
                                 the main reservoir, Paluxy and Tuscaloosa sandstones as well as the 
                                 top seal corresponding to the Claython Chalk. Note that Tuscaloosa
                                 units are discontinuos.}

\newpage

\subsection{Available Data}

Two multicomponent seismic surveys acquired by the RCP in June 2010 and 
August 2011 and shot by Tesla Conquest after the CO$_2$ flooding are used 
for this study (\ref{fig:time_line}).  I will refer from now on as monitor1 
and monitor2 respectively. These surveys covers an area of approximately 
4 mi2 (\ref{fig:rcpArea}). The acquisition parameters for both monitors
are shown in \ref{tab:seispar}. Additionally, a seismic survey before
CO$_2$ injection was shot, however is not relevant for our study because 
was recorded using single component geophones, therefore no multicomponent 
data was registrated.

\csmlongfigure[h!]{time_line}{./chapters/Chapter1_intro/Figures/time_line.pdf}{5in}{
                              Time line of Delhi Field, multicomponent surveys monitor1
                              and monitor2 were shot after CO$_2$ injection.}

\csmlongfigure[h!]{rcpArea}{./chapters/Chapter1_intro/Figures/rcpArea}{6in}{Delhi field, 
                             highlihted in blue the RCP area of study.}


\begin{table}[h!]
\caption{ Seismic acquisition parameters for both monitors \label{tab:seispar}}
\begin{center}
\begin{tabular}{ | l | p{5cm} |}
\hline
Source Parameters             &   \\
\hline
Source interval            & 165 ft  \\
Source line interval           & 495 ft   \\
Source type     & Dynamite   \\
\hline
Receiver Parameters             &   \\
\hline
Receiver interval            & 82.5 ft  \\
Receiver line interval           & 495 ft   \\
Sensors &Vectorseis, single 3 component  270 orientation \\
Record Length & 6 seconds, 2ms sample interval\\
\hline
Grid Parameters             &   \\
\hline
Bin Size & 41.25 by 82.5 ft \\
Azimuth Angle & 196.01 Degrees \\
NW-SE Line (inline) & 1-133 \\
SW-NE Line (xline) & 1-294 \\
\hline
\end{tabular}
\end{center}
\end{table}

The first and second monitor were processed by Sensor Geophysical data. 
Time lapse processing sequence was implemented for PP and PS data by 
using commingled traces. This processing improved the repeatability 
between the surveys, hence cross equalization workflow was not applied to the data. 

Delhi Field contains more than a hundred wells. However, I used the only two wells 
have  both P and S sonics in the RCP area, highlighted in \ref{fig:data}.
Core data is available in the well 159-2 that cored the reservoir interval. \ref{fig:data}
shows the CO$_2$ injection pattern, the triangles and circles, represents the 
injectors and the producers, respectively. The colors indicates the corresponding 
formation for each of the wells. Sixteen injectors are located in the area of study with
the objective of increase the reservoir pressure in the reservoir, 
five corresponding to the Tuscaloosa, eight to the Paluxy formation, and two water
injectors. The producers are located between the injectors.


\csmlongfigure[h!]{data}{./chapters/Chapter1_intro/Figures/data.pdf}{4in}{CO$_2$ 
                         injection pattern, modified from Denbury Resources inc.}


\subsection{Motivation}

The complex stratigraphy of the Tuscaloosa as well as the the Paluxy
sandstones represent a challenge for the  static and dynamic reservoir 
characterization of the reservoir. Previous studies have been oriented
to solve this problem, \citep{ref:nick} generated a 3D geological model
based in cluster analysis. This study was extended by \citep{ref:doug} that 
incorporated seismic inversion and facies analysis to generate an improved
3D geologic and fluid model. \citep{ref:assem} studied the petrophysical properties
as well as the time-lapse response in the monitor1. \citep{ref:holly},
\citep{ref:julio}, and \citep{ref:ishan} studied the time-lapse response in the monitor1
by analyzing amplitudes, impedances and AVO analysis for fluid and lithological 
interpretation, respectively.

I will continue with the dynamic reservoir characterization effort with 
the objective of evaluate the CO$_2$ movement and pressure changes between 
monitor1 and monitor2. I present a workflow based in rock physics modeling 
in order to make an integrated interpretation using well log, core data, 
facies modeling and joint post stack seismic inversion of 4D multicomponent
data.

\subsection{Research Objectives}

\begin{itemize}
\item Evaluate the compartments and connectivity in the Paluxy and Tuscaloosa 
      Formations based on 4D data.  
\item Identification of areas of CO$_2$ saturation and pore pressure
      changes in the Paluxy and Tuscaloosa Formations. 
\item Integrated interpretation of the results by using production data. 
\item Make recommendations to better sweep the reservoir and prevent well failures.
\end{itemize}


\clearpage


   
   

\chapter{Petrophysical properties}


Petrophysical analysis and a rigorous well data quality control (QC)
was done to the logs of wells 140-1 and 159-2 prior to rock physics modeling.
Log analysis is fundamental to understand the petrophysical
properties of the reservoir and for elastic 
parameter estimation.

The analysis started by evaluating the Gamma Ray (GR) log with the cluster
facies analysis done by \cite{ref:doug} which included the integration 
of XRD analysis, mineralogy from point-counting, well logs, core data and seismic data
for depositional environment and facies interpretation. \ref{fig:gr_facies} shows
the GR logs for both wells along with the facies analysis in colors. From this figure, one can 
observe the cleanest facies corresponds to the Tuscaloosa beach/barrier bar, with an
average permeability of 5,200 mD and porosity of 30\%, followed by the Tuscaloosa washover
facies with an average permeability of 3,200 mD and porosity of 32\%. Tuscaloosa fluvial,
and Paluxy facies have an average permeability 700 mD and 1,000 mD respectively. 

Facies modeling showed strong correlation with gamma ray;
therefore, with the lithology. In this chapter, facies
correlation with density, sonic logs and elastic parameters
will be discussed.  


\csmlongfigure[h!]{gr_facies}{./chapters/Chapter2_WellQC/Figures/gr_facies.pdf}{5in}{
Gamma ray logs for wells 159-2 and 140-1. Colors indicate the facies
interpretation done by \cite{ref:doug}. Tuscaloosa beach/barrier bar represents 
the cleanest sandstone while Tuscaloosa fluvial is a shaly sandstone. Black color indicates
the non-reservoir rocks.} 
 

\subsection{Densities and velocities QC}

Caliper log data can indicate washout zones or areas where the borehole is out
of gage. The washout zones may create
anomalous data that need to be corrected for borehole rugosity. Density is one
 of the most affected logs because the measuring tool has to be in contact with 
the wellbore wall \citep{ref:tad}. 
\ref{fig:159-2_QC,fig:140-1_QC} and \ref{fig:140-1_QC} show the caliper and density logs. One can observe there are no washout zones that affecting the density measurements. 
The caliper log is well behaved through the formation, therefore density logs were
 not edited at this stage.


\begin{sidewaysfigure}
\centering
\scalebox{0.5}
{\includegraphics[width=15in]{./chapters/Chapter2_WellQC/Figures/159-2_QC.pdf}}
\caption{Caliper, gamma ray, density and sonic logs of well 159-2. The black color indicates
the observed logs while red color indicates pseudo logs. Gasmmann's equation was used
to generate the density pseudo log. Castagna's equations were used to calculate compressional
and shear velocity.} 
\label{fig:159-2_QC}
\end{sidewaysfigure}

\begin{sidewaysfigure}
\centering
\scalebox{0.5}
{\includegraphics[width=15in]{./chapters/Chapter2_WellQC/Figures/140-1_QC.pdf}}
\caption{Caliper, gamma ray, density and sonic logs of well 140-1. The black color indicates
the observed logs while red color indicates pseudo logs. Gasmmann's equation was used
to generate the density pseudo log. Castagna's equations were used to calculate compressional
and shear velocity.} 
\label{fig:140-1_QC}
\end{sidewaysfigure}


Gardner's equation was used to generate pseudo density logs from compressional velocities
\citep{ref:gardner}:

\begin{equation}
\rho=\alpha V_{p}^{\beta},
  \label{eq:gardner}
\end{equation}

where $\alpha$ is 0.31, $\beta$ is 0.25, $V_{p}$ units are m/s and $\rho$ units are
g/cm$^3$. \ref{fig:159-2_QC} and \ref{fig:140-1_QC} show in red the pseudo density log
and in black the observed log. Discrepancies are observed because Gardner's empirical equation
is more reliable  for well cemented and brine saturated rocks which is not the case 
for Delhi Field rocks. Hence, Gardner's equation is not appropriated for Delhi, calibration of the 
coefficients is recommended in order to have better results, especially in the shale zones. 

Castagna's empirical equations for compressional and shear velocity estimation
were used as a QC tool \citep{ref:castagna}. These equations relate the porosity and the clay 
content for brine saturated rocks:

\begin{equation}
V_{p}=5.81-9.42\phi-2.2C,
  \label{eq:castagna1}
\end{equation}

\begin{equation}
V_{s}=3.89-7.07\phi-2.04C,
  \label{eq:castagna2}
\end{equation}

where $\phi$ is the porosity and $C$ is the clay content. The porosity log used for 
this calculation corresponds to Magnetic-Resonance-Imaging-Log (MRIL). The
tool generates a static magnetic field that magnetizes the spin axes of the protons that 
originally are align with the earth's magnetic field. The tool generates spin echoes
and records the amplitude of the spin echo-train which is proportional to the hydrogen 
protons, therefore to the porosities \citep{ref:george}. Clay content was calculated from
the GR logs.

\ref{fig:159-2_QC} and \ref{fig:140-1_QC} show in red color the pseudo logs and in black the
observed logs. Both wells indicate that predictions have higher frequencies than the
observed values. Lower frequencies show the same behavior in the sandstones, therefore the clay and porosity models
values can be used to model the shear and compressional velocities. However, these modeles overestimate the 
velocities in the shale zones. This behavior is expected for rocks composed primarily of clay \citep{ref:castagna}.  

\subsection{Dry bulk modulus calculation}

One of the most important elastic parameters for rock physics modeling 
is the dry frame bulk modulus of the rock ($K_{dry}$). $K_{dry}$ when calculated by 
using Gassmann's equation is usually noisy, and may contain negative poisson
ratio values \citep{ref:tad}. For this reason, \cite{ref:krief} simple heuristic model
was used to calculate $K_{dry}$:

\begin{equation}
K_{dry}=G \frac{K_{0}}{G_{0}},
  \label{eq:kdry}
\end{equation}

where $K_{0}$ and $G_{0}$ are the bulk and the shear modulus respectively of the mineral matrix. These moduli
were calculated using the Voigt-Reuss-Hill average that is the average between Voigt ($M_{v}$)
and Reuss ($M_{r}$) bounds:

\begin{equation}
M_{H}= \frac{M_{v}+M_{r}}{2}=\frac{1}{2}[\sum_{i=1}^{N}f_{i}M_{i}+(\sum_{i=1}^{N}f_{i}M_{i}^{-1})^{-1})],
  \label{eq:hill}
\end{equation}

the term $M_{i}$ is the moduli and $f_{i}$ is the volume fraction \citep{ref:handbook}. For this modeling,
I assumed the rocks where composed of quartz and clay. This assumption is consistent with the XRD results 
for core 159-2 (see appendix \ref{app:xrds}, \ref{fig:xrd}). \ref{tab:moduli}  shows the bulk and shear modulus values used. 
 
\begin{table}[h!]
\caption{ Bulk and shear modulus used for $K_{dry}$ calculation \label{tab:moduli}}
\begin{center}
\begin{tabular}{ | l | l |l|}
\hline
Mineral      & $K$ (Gpa) & $G$ (Gpa)  \\
\hline
Quartz & 37.0 & 44.0 \\
\hline
Clay & 25.0 &9.0 \\
\hline
\end{tabular}
\end{center}
\end{table}


\ref{fig:159-2QC2} and \ref{fig:140-1QC2} show the calculated $K_{dry}$. The values
range between 3.0 and 5.0 Gpa. The lowest values are observed in the sandstones
with the exception of the Tuscaloosa beach/barrier facies which indicates
 possible cementation.

\csmlongfigure[h!]{159-2QC2}{./chapters/Chapter2_WellQC/Figures/159-2QC2.pdf}{5in}{
Gamma ray, $V_{p}/V_{s}$ ratio of the dry rock frame and $K_{dry}$ of the dry rock frame of well 159-2.}

\csmlongfigure[h!]{140-1QC2}{./chapters/Chapter2_WellQC/Figures/140-1QC2.pdf}{5in}{
Gamma ray, $V_{p}/V_{s}$ ratio of the dry rock frame and $K_{dry}$ of the dry rock frame of well 140-1.}

$V_{p}$/$V_{s}$ dry ratio is another QC shown in \ref{fig:159-2QC2} and \ref{fig:140-1QC2}. One can observe that the 
lowest $V_{p}$/$V_{s}$ ratio is consistent with the cleanest facies and that  
$V_{p}$/$V_{s}$ ratio is a lithology discriminator between sands and shales. Higher 
$V_{p}$/$V_{s}$ ratios indicate high clay content. 

\newpage
\subsection{Fluid substitution}

Rock saturated bulk modulus was calculated using the isotropic form of Gasmmann's equation 
\citep{ref:gasmmann}:
\begin{equation}
K_{sat}= K_{dry}+\frac{(1-K_{dry}/K_{0})^{2}}{\phi/K_{fl}+(1-\phi)/K_{0}-K_{dry}/K_{0}^2},
  \label{eq:gasmmann}
\end{equation}

where $K_{dry}$ is the dry bulk frame modulus calculated in Equation \ref{eq:kdry}, $K_{0}$ 
is the bulk modulus of the mineral matrix, $\phi$ is the total porosity and $K_{fl}$ is 
the fluid bulk modulus calculated by using Hill's average (\ref{eq:hill}). The oil saturation
values were obtain by the MRIL logs. The reader is referred to \cite{ref:george} for
a review of nuclear magnetic logs (NMR).

The fluid properties used for the modeling are shown in \ref{tab:fluid}. The moduli
of the fluid were calculated using Batzle and Wang spreadsheet and are shown in \ref{tab:flag}. 


\begin{table}[h!]
\caption{Fluid properties used for $K_{sat}$ modeling \label{tab:fluid}}
\begin{center}
\begin{tabular}{ | l | l |}
\hline
Temperature      & 137 F  \\
\hline
Pressure      & 1450 PSI  \\
\hline
Oil Gravity & 43.9 API \\
\hline
Gas Gravity & 0.65 \\
\hline
Salinity & 80,000 \\
\hline
\end{tabular}
\end{center}
\end{table}


\begin{table}[h!]
\caption{Brine and oil moduli \label{tab:flag}}
\begin{center}
\begin{tabular}{ | l | l |l|}
\hline
Fluid type & $\rho$ ($g/cm^3$)& K (GPa) \\
\hline
Brine      &  1.03 & 2.81  \\
\hline
Oil      & 0.69 & 0.82   \\
\hline
\end{tabular}
\end{center}
\end{table}

$V_{p}$ was calculated as another QC tool by using:
 
\begin{equation}
V_{p}=\sqrt{\frac{K_{sat}+4/3 G_{sat}}{\rho_{sat}}}, 
  \label{eq:gassmann}
\end{equation}

where $K_{sat}$ is the bulk modulus of the rock calculated in \ref{eq:gasmmann}, $G_{sat}$ is the shear
modulus calculated from the logs and $\rho_{sat}$ is the rock density.


\ref{fig:159-2_vp} and \ref{fig:140-1_vp} show the oil saturation log, the $K_{sat}$, 
the $V_{p}$ observed and the $V_{p}$ calculated log. Well 140-1 in \ref{fig:140-1_vp}  shows good correlation
 between the observed and calculated $V_{p}$ which indicates consistency between the 
$K_{dry}$, oil saturation log and the recorded sonic logs. Well 159-2 in  \ref{fig:159-2_vp} 
does not show good correlation between the observed and calculated $V_{p}$. Calculated $V_{p}$ 
is underestimating the sonic log values, this behavior is possibly related to invasion effects of 
the water-based mud which could be penetrating the formation with brine. The invasion effect
can be seen in \ref{fig:invasion} where the shallow and deep resistivity logs show discrepancies
in the oil saturated zones.

\cite{ref:han} suggest that velocities and densities affected by invasion are difficult to correct.
Incorrect values can yield wrong reflectivity models producing misties with the seismic. 
I use the velocity and density pseudo-logs generated in this chapter for 
the rock physics modeling and the seismic ties. 


\csmlongfigure[h!]{159-2_vp}{./chapters/Chapter2_WellQC/Figures/159-2_vp.pdf}{4.5in}{
Gamma ray, oil saturation, and $V_{p}$ of well 159-2. Black color indicates the recorded sonic. 
Red color indicates the pseudo-log.}

\csmlongfigure[h!]{140-1_vp}{./chapters/Chapter2_WellQC/Figures/140-1_vp.pdf}{4.5in}{
Gamma ray, oil saturation, and $V_{p}$ of well 159-2. Black color indicates the recorded sonic. 
Red color indicates the pseudo-log.}

\csmlongfigure[h!]{invasion}{./chapters/Chapter2_WellQC/Figures/invasion.pdf}{4.5in}{10 inch and 90
inch resistivity logs of well 159-2. Note how the resistivity values changes in the 
oil saturated zones due to invasion effects.}

\subsection{QC Crossplot}

$V_{p}/V_{s}$ ratio vs acoustic impedance is a crossplot commonly used in rock physics
because we can evaluate well-log data and seismic inversion data \citep{ref:avseth}.
\ref{fig:ip_vpvs} shows the crossplot, the color scale is the clay volume obtain
from GR logs. A clear separation between sandstones and shales is observed.
This qualitative result is encouraging for the rock physics modeling and 
joint inversion framework. The next chapter will provide a calibration of 
rock physics models based on logs with the objective to do a quantitave interpretation
of the data.


\csmlongfigure[h!]{ip_vpvs}{./chapters/Chapter2_WellQC/Figures/ip_vpvs.pdf}{6in}{
$V_{p}/V_{s}$ ratio vs acoustic impedance for wells 140-1 and 159-2}


\newpage\mbox{}\newpage

\chapter{Dynamic rock physics modeling}

The geological processes that affects siliciclastic sediments and rock porosities have been 
studied by many authors (references). Trends in velocity 
porosity relations can indicate diagenetic or depositional processes that 
give information about the rock fabric \citep{ref:avseth}. \ref{fig:processes} shows 
a summary of the processes that controls porosity on the clastic rocks. Steep slopes
indicates stiffness, compaction and cementation and flatter slopes corresponds
to sorting and clay content. Therefore, evaluating the data in the rock physics framework
helps to understand the reservoir nature.
 
\csmlongfigure[h!]{processes}{./chapters/Chapter3_Rock_physics/Figures/processes}{4.0in}{
Depositional and diagenetic processes that affects sandstones porosity. Taken from
\cite{ref:avseth} }

In this study I use rock physics modeling as a tool to integrate  core, petrophysical, 
XRD, facies data and 4D joint seismic inversion to make a quantitative
interpretation of Delhi Field reservoir changes due to CO$_2$ injection.
For the static rock physics modeling I follow the workflow suggested by \cite{ref:avseth}. This 
consist on a hybrid  approach that uses theoretic, heuristic and empirical
models built for high porosity sandstone reservoirs. The used models are best known as friable
sand, contact cement and constant cement models. 

For the dynamic rock physics modeling I follow the workflow suggested in \cite{ref:dynamic} that
relates relative changes in $V_{p}/V{s}$ ratio and acoustic impedance to pore pressure
and fluid changes. Then, I extend this qualitative analysis into quantitative calculating
pore pressure maps for each facies. The pore pressure calculation is important property for maximize
hydrocarbon recovery, for reservoir management and for drilling safety.

The detail workflow of the modeling is shown in \ref{fig:workflow}.
The subsection of the chapters corresponds to the colors of the workflow

\ref{fig:workflow}
\csmlongfigure[h!]{workflow}{./chapters/Chapter3_Rock_physics/Figures/workflow2}{5.0in}{logs}


\subsection{Rock physics models}
bla

\subsection{Model calibration}
bla
bla \ref{fig:theoryModel}
\csmlongfigure[h!]{theoryModel}{./chapters/Chapter3_Rock_physics/Figures/theoryModel}{5in}{logs}


bla \ref{fig:model_calibration}
\csmlongfigure[h!]{model_calibration}{./chapters/Chapter3_Rock_physics/Figures/model_calibration}{5in}{logs}

bla \ref{fig:model_faciesQ85}
\csmlongfigure[h!]{model_faciesQ85}{./chapters/Chapter3_Rock_physics/Figures/model_faciesQ85}{7in}{logs}

bla \ref{fig:thinTW}
\csmlongfigure[h!]{thinTW}{./chapters/Chapter3_Rock_physics/Figures/thinTW}{5in}{logs}

bla \ref{fig:facies_modelingP}
\csmlongfigure[h!]{facies_modelingP}{./chapters/Chapter3_Rock_physics/Figures/facies_modelingP}{5in}{logs}

bla \ref{fig:thinTB}
\csmlongfigure[h!]{thinTB}{./chapters/Chapter3_Rock_physics/Figures/thinTB}{5in}{logs}

bla \ref{fig:facies_modelingTB}
\csmlongfigure[h!]{facies_modelingTB}{./chapters/Chapter3_Rock_physics/Figures/facies_modelingTB}{5in}{logs}

bla \ref{fig:thinTF}
\csmlongfigure[h!]{thinTF}{./chapters/Chapter3_Rock_physics/Figures/thinTF}{5in}{logs}

bla \ref{fig:facies_modelingTF}
\csmlongfigure[h!]{facies_modelingTF}{./chapters/Chapter3_Rock_physics/Figures/facies_modelingTF}{5in}{logs}

\newpage
\subsection{Static Rock Physics template}

bla \ref{fig:staticModelP}
\csmlongfigure[h!]{staticModelP}{./chapters/Chapter3_Rock_physics/Figures/staticModelP}{7in}{logs}



\ref{tab:modeling}
\begin{table}[h]
\caption{ Modeling parameters \label{tab:modeling}}
\begin{center}
\begin{tabular}{ | l | l | l|  l | l | l|}
\hline
  Facies            &  n & Quartz & Clay & Siderite & P$eff$ (Mpa)  \\
\hline
Tusc Fluvial        &                 8.64 &   0.7  &  0.2 &     0.1  &  12.5  \\
\hline
Tusc Beach/Barrier  & 12.96                &   0.97  &  0.03 &     0.0  &  12.5  \\
\hline
Tusc Washover/Transitional and Paluxy  & 10.36    &   0.85  &  0.15 &     0.0  &  12.5  \\
\hline
\end{tabular}
\end{center}
\end{table}



%\subsubsection{Static rock physics template}
%\subsubsection{Dynamic rock physics template}

\chapter{Inversion}

A key benefit of having both P-P and P-S data is to use them together to more tightly constrain the inversion process which provides reservoir layer properties.
The objective of the inversion was to  transform an interface property into a layer property.  According to Tarantola (2005), �the inverse problem consists of using the actual result of some measurements to infer the values of the parameters that characterize the system�. In this case, the measurements were the P-P and P-S post stack data and the well logs. The parameters that characterize the system are the acoustic impedance, the shear impedance and the density volume. The studied reservoir is a layer of sandstone filled with oil, the sandstone has high porosity and a lower impedance than the surrounding shales. To generate acoustic impedance, shear impedance and density volume, I performed a P-P and P-S Joint Inversion (Hampson and Russell, 2008b) (\ref{fig:inversion_flow}).  To perform this inversion, I conducted a well tie to create P-P and P-S time-depth curves. Then I converted the P-S data from P-S time to P-P time domain. The second step was to build a joint inversion low-frequency model. The model was created using the well logs and seismic horizons. After some tests, I use only one well in the model creation, the Well 01D, the only well that penetrates the reservoir and has P-wave velocity and S-wave velocity. The low-frequency model created for the joint inversion is done with the density, S-wave and P-wave logs. The input to the inversion were the  P-P and P-S wavelets extracted in P-P time, the P-P and P-S seismic volumes and the joint inversion low-frequency model. The main outputs were the acoustic impedance (Zp), the shear impedance (Zs) and the density volume.






\begin{figure}[hbtp]
	\begin{center}
	\includegraphics[width=0.9\linewidth]{./chapters/Chapter3_Inversion_latex/Figures/Fig1_inversion_procedures}
			\caption[P-P and P-S joint inversion procedures  (Hampson and Russell (2008b).]{P-P and P-S joint inversion procedures (Hampson and Russell (2008b). The input data were the P-P and P-S post stack data, the wavelets in P-P time extracted from the P-P and the P-S data, and the low frequency model build after the well logs (P-wave velocity, S-wave velocity and density) and seismic horizons. The main outputs were the acoustic impedance (Zp), the shear impedance (Zs) and the density volume.}
			\label{fig:inversion_flow}
		\end{center}
	\end{figure}


\subsection{Model Building}

The model based P-P, P-S joint inversion requires a low frequency model to run, but the seismic data does not have low frequency. The generation of the low-frequency model is the most important step of the inversion process. In order to achieve reasonable outputs from the inversion process, I tried a range of values for several tests to create the the low-frequency model.
	
	Test 1: I did the horizon matching between the top of the carbonate in the P-P and in the P-S seismic volumes. The top of the carbonate map has some uncertainty, and the pick was difficult to make in certain locations. This uncertainty in the top of the carbonate interpretation, in both P-P and P-S seismic volumes, created some distortions in the P-P and P-S velocity model. Consequently, the results from the inversion using this model has all the errors associated with the P-P and P-S top of carbonate interpretations problems.
	
	Test 2: As the horizon match is not a mandatory step in the inversion process, I skipped it. The results were good near the well. Away from the well I did not have a way to match the P-P and P-S data and the impedance results were less reliable.
	
	Test 3: I used just one horizon to match the P-P and P-S seismic data. The horizon used was the top of the pebbly zone. In the model creation, I used both top of the pebbly zone and the top of the carbonate platform. The wells used were the ones with higher synthetic and seismic correlation coefficient for both P-P and PS data. I used these three wells for the low frequency model generation.
	
	Test 4: I used only one horizon to match the P-P and P-S seismic data. The horizon used was the top of the reservoir. In the model creation, I used the top of the pebbly zone, the top of the reservoir and the top of the carbonate. I used only one well (Well 01D) in the model, I chose it because this is the only well that has P-wave and S-wave logs that penetrate the oil reservoir. The result of this modeling is a model of Vp, Vs and density (\ref{fig:density_model}).

	A joint inversion was done using each one of the low-frequency model tested. Based on the results of the inversion, the Test 4 was interpreted to be the preferred model.

\begin{figure}[hbtp]
	\begin{center}
	\includegraphics[width=0.9\linewidth]{./chapters/Chapter3_Inversion_latex/Figures/Fig2_Model_July27_w01D_DEN_Inline1334}
			\caption[Crossline 591, density model.]{Crossline 591, density model. This model was created from the well logs and the main seismic horizons.}
			\label{fig:density_model}
		\end{center}
	\end{figure}



\subsection{P-P and P-S well tie}

I used the well log markers interpreted from Petrobras geologists as a guide to tie the well and seismic. Most of the well logs have the pebbly zone interpreted. I tied the top of the pebbly zone seismic horizon interpreted in time, with the pebbly zone well marker in the log interpreted in depth. The selection of the three wells was based mainly on the presence of P-wave and S-wave velocity logs, and secondarily on a good correlation with the seismic and well. The wells selected were 01D (the only one with an S-wave velocity log in the reservoir) and the wells 03 and 04 (vertical wells with good correlation coefficient   between the seismic and the logs). 

The P-P and P-S well ties require a wavelet which can be extracted from the data. I chose, however to use Ricker wavelets in the synthetic creation as they were very similar to the extracted wavelets. Different frequencies were tested to see which one was most similar to the seismic data in the reservoir level (Figures 3.4, 3.5, 3.6, 3.7 and 3.8). 
	Well 01D was a key well to my inversion. At Well 01D, the P-P synthetic seismic has a correlation coefficient of 81 percent with the P-P seismic data in a window between 2600 ms and 3075 ms (\ref{fig:synth_01D_PP}). The wavelet used was a 20 Hz Ricker   in the P-P synthetic and a 13 Hz Ricker on the P-S synthetics. The P-S synthetic has a correlation coefficient of 65 percent with the P-S seismic data in a window between 4550 ms and 5250 ms (\ref{fig:synth_01D_PS}).  

\begin{figure}[hbtp]
	\begin{center}
	\includegraphics[width=0.9\linewidth]{./chapters/Chapter3_Inversion_latex/Figures/Fig3_synth_01_PP}
			\caption[P-P synthetic-field data match at well 01D. ]{P-P synthetic-field data match at well 01D. In blue is the synthetic seismic generated from the P-wave and density logs. In red the trace from the seismic extracted around the well and repeated five times. In black are the nine traces nearest to the well. Note the similarities between the synthetic and the traces around the well; the correlation coefficient is 81 percent. The wavelet used was a 20 Hz Ricker  wavelet.}
			\label{fig:synth_01D_PP}
		\end{center}
	\end{figure}



\begin{figure}[hbtp]
	\begin{center}
	\includegraphics[width=0.9\linewidth]{./chapters/Chapter3_Inversion_latex/Figures/Fig4_synth_01_PS}
			\caption[P-S synthetic-field data match at well 01D.]{P-S Synthetic-field data match at well 01D. In blue is the synthetic seismic generated from the S-wave and density logs. In red is the trace from the seismic extracted around the well and repeated five times, and in black are the nine traces from the seismic data nearest to the well. Note the similarities between the synthetic and the traces around the well; the correlation coefficient is 65 percent. The wavelet used was a 13 Hz Ricker wavelet. }
			\label{fig:synth_01D_PS}
		\end{center}
	\end{figure}



 The P-P synthetic from well 03 (the wavelet used was a 20 Hz Ricker) has a correlation coefficient of 71 percent with the P-P seismic data in a window between 2500 ms and 2800 ms (\ref{fig:synth_03_PP}). The P-S synthetic has a correlation coefficient of 83 percent with the P-S seismic data in a window between 4400 ms and 4950 ms (\ref{fig:synth_03_PS})  with a 10 Hz Ricker wavelet.  


\begin{figure}[hbtp]
	\begin{center}
	\includegraphics[width=0.9\linewidth]{./chapters/Chapter3_Inversion_latex/Figures/Fig5_synth_03_PP}
			\caption[P-P synthetic-field data match at well 03. ]{P-P synthetic-field data match at well 03. In blue is the synthetic seismic generated from the P-wave and density logs. In red is the trace from the seismic extracted nearest the well and repeated five times, and in black are the nine traces from the seismic data nearest to the well. Note the similarities between the synthetic and the traces around the well; the correlation coefficient is 71 percent. The wavelet used was a 20 Hz Ricker wavelet.}
			\label{fig:synth_03_PP}
		\end{center}
	\end{figure}
	
	
\begin{figure}[hbtp]
	\begin{center}
	\includegraphics[width=0.9\linewidth]{./chapters/Chapter3_Inversion_latex/Figures/Fig6_synth_03_PS}
			\caption[P-S synthetic-field data match at well 03. ]{P-S synthetic-field data match at well 03. In blue is the synthetic seismic generated from the S-wave and density logs. In red is the trace from the seismic extracted nearest the well and repeated five times, and in black are the nine traces from the seismic data nearest to the well. Note the similarities between the synthetic and the traces around the well; the correlation coefficient is 76 percent. The wavelet used was a 10 Hz Ricker.}
			\label{fig:synth_03_PS}
		\end{center}
	\end{figure}	


The third well was well 04.  The P-P synthetic has a correlation coefficient of 83 percent with the P-P seismic data in a window between 2500 ms and 2950 ms (\ref{fig:synth_04_PP}). In the P-P synthetic the wavelet used was a 15 Hz Ricker.  In the P-S synthetic the wavelet used was a 10 Hz Ricker. The P-S synthetic has a correlation coefficient of 80 percent with the P-S seismic data in a window between 4400 ms and 5100 ms (\ref{fig:synth_04_PS}) .  


\begin{figure}[hbtp]
	\begin{center}
	\includegraphics[width=0.9\linewidth]{./chapters/Chapter3_Inversion_latex/Figures/Fig7_synth_04_PP}
			\caption[P-P synthetic-field data match at well 04. ]{P-P synthetic-field data match at well 04. In blue is the synthetic seismic generated from the S-wave and density logs. In red is the trace from the seismic extracted nearest to the well, and in black are the nine traces from the seismic data nearest to the well. Note the similarities between the synthetic and the traces around the well; the correlation coefficient is 83 percent. The wavelet used was a 15 Hz Ricker.}
			\label{fig:synth_04_PP}
		\end{center}
	\end{figure}

\begin{figure}[hbtp]
	\begin{center}
	\includegraphics[width=0.9\linewidth]{./chapters/Chapter3_Inversion_latex/Figures/Fig8_synth_04_PS}
			\caption[P-S synthetic-field data match at well 04. ]{P-S synthetic-field data match at well 04. In blue is the synthetic seismic generated from the S-wave and density logs. In red is the trace from the seismic extracted nearest to the well, and in black are the nine traces from the seismic data nearest to the well. Note the similarities between the synthetic and the traces around the well; the correlation coefficient is 80 percent. The wavelet used was a 10 Hz Ricker.}
			\label{fig:synth_04_PS}
		\end{center}
	\end{figure}



\subsection{P-P and P-S wavelets}
I extracted a statistical P-P wavelet from the P-P seismic data (\ref{fig:PP_wavelet}) and a statistical P-S wavelet from the P-S seismic data in the P-P time domain (\ref{fig:PS_wavelet}) to be used as input for the P-P and P-S post-stack joint inversion (Hampson and Russell (2008b)).

\begin{figure}[hbtp]
	\begin{center}
	\includegraphics[width=0.9\linewidth]{./chapters/Chapter3_Inversion_latex/Figures/Fig9_wavelet_PP}
			\caption[P-P wavelet extracted from the P-P seismic data in a time window between 2300 ms and 3200 ms.]{P-P wavelet extracted from the P-P seismic data in a time window between 2300 ms and 3200 ms. In the left, the wavelet in time and in the right the wavelet amplitude spectrum.}
			\label{fig:PP_wavelet}
		\end{center}
	\end{figure}

\begin{figure}[hbtp]
	\begin{center}
	\includegraphics[width=0.9\linewidth]{./chapters/Chapter3_Inversion_latex/Figures/Fig10_wavelet_PS}
			\caption[P-S wavelet extracted from the P-S seismic data in a P-P time domain.]{P-S wavelet extracted from the P-S seismic data in a P-P time domain. The time window is between 2300 ms and 3200 ms. In the left the wavelet in time and in the right the amplitude spectrum.}
			\label{fig:PS_wavelet}
		\end{center}
	\end{figure}


\subsection{Inversion Parameters}
The joint inversion P-P time range was between 2000ms to 3500ms.\\
The input data for this inversion were:\\
	-	the P-P and P-S migrated post-stack volumes\\
	-	Vp, Vs and density models\\
	-	P-wave and S-wave logs\\
	-	Regression coefficients that define the relationships between Zp and density, and between Zp and Zs.\\

The regression coefficients control the background relationship for the inversion parameters and are determined from the cross plots (\ref{fig:crossplots}) of the natural logarithms of the impedances Zp and Zs and the density (as derived from the well data). The coefficients are based on these equations, which are used to reduce the non-uniqueness of the inversion:

\begin{equation}
ln(Z_S)=k \ ln(Z_P) + k_c +  \Delta L_S 
\end{equation}

\begin{equation}
ln(\rho)=m \ ln(Z_P) + m_c +  \Delta L_{\rho}
\end{equation}

where,
k = angular coefficient of the linear curve from the cross-plot between Zs and Zp.\\
m = angular coefficient of the linear curve from the cross-plot between density and Zp.\\
L = distance of the points in the cross-plots from the linear curve that best fit the data.\\


\begin{figure}[hbtp]
	\begin{center}
	\includegraphics[width=0.9\linewidth]{./chapters/Chapter3_Inversion_latex/Figures/Fig12_Zs_Zp_crossplot_JointInversionParameters_July27}
			\caption[Cross-plots  of the natural logarithms of the impedances Zp and Zs and the density (as derived from the well 01D).]{Cross-plots  of the natural logarithms of the impedances Zp and Zs and the density (as derived from the well 01D). }
			\label{fig:crossplots}
		\end{center}
	\end{figure}


\subsection{Inversion Results}


The P-P and P-S post-stack joint inversion assisted the reservoir characterization. The shear-impedance attribute map (\ref{fig:map_Zs}) extracted in a 30 ms window below the top of the reservoir has better resolution than the P-S amplitude seismic data and the inverted density improved the image in the reservoir lower zone. The minimum acoustic impedance attribute map (\ref{fig:map_Zp}), extracted in a 30 ms window below the top of the reservoir is similar to the minimum amplitude or most negative amplitude map extracted in  a 10 ms centered window at the top of the reservoir from the P-P seismic data. The size of attribute extraction windows are different because the extraction of an interface property requires a window size proportional to the size of the seismic resolution and to extract a layer property I need a window the size proportional to the size of the layer. 

The inversion results can be thought of as the transformation of amplitudes, caused by interface properties into a layer properties calibrated with the well logs. An important advantage of the layer properties from the inversion is that it has values that can be correlated with the well logs and used to populate a geologic model. 


The acoustic impedance (\ref{fig:Inline_1334_Zp}), the shear impedance (\ref{fig:Inline_1334_Zs}), and the density volume (\ref{fig:Inline_1334_Den}) are consistent with the well logs. The reservoir has lower acoustic impedance, lower shear impedance and lower density than the surrounding shale.  


\begin{figure}[hbtp]
	\begin{center}
	\includegraphics[width=0.9\linewidth]{./chapters/Chapter3_Inversion_latex/Figures/Fig14_map_Simpedance_min25msbelow}
			\caption[Inverted shear-impedance attribute map, extracted in a 25 ms window below the top of the reservoir.]{Inverted shear impedance attribute map, extracted in a 25 ms window below the top of the reservoir.}
			\label{fig:map_Zs}
		\end{center}
	\end{figure}



\begin{figure}[hbtp]
	\begin{center}
	\includegraphics[width=0.9\linewidth]{./chapters/Chapter3_Inversion_latex/Figures/Fig13_map_Pimpedance_min25msbelow}
			\caption[Inverted acoustic-impedance attribute map, extracted in a 25 ms window below the top of the reservoir.]{Inverted acoustic impedance attribute map, extracted in a 25 ms window below the top of the reservoir.}
			\label{fig:map_Zp}
		\end{center}
	\end{figure}
  


\begin{figure}[hbtp]
	\begin{center}
	\includegraphics[width=1.0\linewidth]{./chapters/Chapter3_Inversion_latex/Figures/Fig15_inline1334_Pimpedance_zoomwell01D}
			\caption[ Acoustic impedance, inline 1358. ]{ Acoustic impedance, inline 1358 with acoustic impedance well log from well 11.}
			\label{fig:Inline_1334_Zp}
		\end{center}
	\end{figure}

\begin{figure}[hbtp]
	\begin{center}
	\includegraphics[width=1.0\linewidth]{./chapters/Chapter3_Inversion_latex/Figures/Fig16_inline1334_Simpedance_zoomwell01D}
			\caption[ Shear impedance, inline 1330.]{ Shear impedance, inline 1330 with shear impedance and density well log from well 01D.}
			\label{fig:Inline_1334_Zs}
		\end{center}
	\end{figure}


\begin{figure}[hbtp]
	\begin{center}
	\includegraphics[width=1.0\linewidth]{./chapters/Chapter3_Inversion_latex/Figures/Fig16_density}
			\caption[ Density , crossline 399.]{ Density, crossline 399 with density well log from well 05D.}
			\label{fig:Inline_1334_Den}
		\end{center}
	\end{figure}
	

\subsection{Fluid replacement modeling}

Fluid substitution modeling was performed using the Gassmann's equation. I studied the contrast in the shear impedance and in the acoustic impedance between the reservoir and the overlying shale. The well logs used were the P-wave velocity , the S-wave velocity and the density. The well used was the well 01D. I calculated the P-wave and the S-wave velocities and the density in the reservoir were calculeted using three scenarios: (1) 80 percent filled with oil, 20 percent with brine, (2) 80 percent filled with brine, 20 percent filled with oil, and (3) 80 percent filled with gas, 20 percent filled with brine (\ref{fig:FRM_res}). In the overlying shale  values were held constant with the shale saturated with 100 percent with brine (\ref{fig:FRM_shale}). The fluid and reservoir properties used in the modeling are shown in Table 3.1. Using the three fluid replacement scenarious, the acoustic and shear impedance variations were calculated between the shale and the reservoir. The results are shown in \ref{fig:FRM_table2}:


\begin{figure}[hbtp]
	\begin{center}
	\includegraphics[width=1.0\linewidth]{./chapters/Chapter3_Inversion_latex/Figures/FRM_res}
			\caption[Fluid replacement modeling in the reservoir.]{Fluid replacement modeling in the reservoir.}
			\label{fig:FRM_res}
		\end{center}
	\end{figure}
	
\begin{figure}[hbtp]
	\begin{center}
	\includegraphics[width=1.0\linewidth]{./chapters/Chapter3_Inversion_latex/Figures/FRM_shale}
			\caption[Brine saturated overlying shale, density, S-wave and P-wave velocity values.]{Brine saturated overlying shale, density, S-wave and P-wave velocity values.}
			\label{fig:FRM_shale}
		\end{center}
	\end{figure}


\begin{table} \label{table_fluid_properties} 
\begin{center} 
\begin{tabular}{| c | c | } 
\hline 
Temperature                & 89 $^o$ degree  Celsius     \\ 
\hline 
Oil gravity         &  29 API       \\ 
\hline 
Gas gravity           & 0.79        \\ 
\hline 
Gas-oil-ration   &  421 scft/sbbl       \\ 
\hline
Salinity   &  100,000 ppm  \\
\hline
Pressure   &    3982 psi  \\
\hline 
\end{tabular} 
\caption{Fluid properties used in the fluid replacement modeling.} 
\end{center} 
\end{table}





\begin{figure}[hbtp]
	\begin{center}
	\includegraphics[width=0.9\linewidth]{./chapters/Chapter3_Inversion_latex/Figures/FRM_table2}
			\caption[Fluid replacement modeling results table.]{Fluid replacement modeling results table.}
			\label{fig:FRM_table2}
		\end{center}
	\end{figure}

 The results from the fluid replacement model matches the seismic results. The P-P amplitude shows a strong trough in the top of the oil reservoir indicating that the acoustic impedance contrast when the reservoir is oil saturated  is strong, the acoustic impedance is very sensible to the fluid saturation. The P-S amplitude has a weak trough in the top of the reservoir indicating that the shear velocity is not sensible to fluid content. In the inversion results the reservoir has a lower acoustic impedance and a lower shear impedance than the overlying shales matching the results from the modeling done above.




 


\chapter{Interpretation}
the interpretation

\chapter{Conclusions and recommendations}
In this chapter,





%\csmlongfigure{Strongbad}{figures/strongbad}{1in}{A world-class hero}{ of awesomeness~\cite{ref:Wikipedia}.}

\backmatter

\bibliography{thesis}
%\printbibliography % <-- For using biblatex instead of natbib or the built-in bibliography utility

%%% Parts of a Thesis - Back Matter - Selected Bibliography (optional)
%\cleardoublepage
%\begin{selected-bibliography}
% Your selected bibliogrpahy would go here, a page break might also be necessary above.
%\end{selected-bibliography}

%%% Parts of a Thesis - Back Matter - Appendices (if applicable)
\appendix{XRDs results for core 159-2}\label{app:xrds}
%\ref{tab:encoding} shows how several symbols appear in the rendered document.
\csmlongfigure[h!]{xrd}{xrd.png}{5in}{XRDs results for core 159-2}



%\appendix{Special Coolness}

%Insert ice cubes here (\ref{lst:hello-world}).

%\lstinputlisting[language=Matlab,label={lst:hello-world},caption={A MATLAB ``Hello World`` Example}]{matlab_code.m}


\end{document}
