\chapter{Introduction}
Delhi Field is a case of study of the Reservoir Characterization Project (RCP)
since 2009, it is an excellent example for an enhanced oil recovery (EOR) 
active monitoring field with 4D multicomponent seismic technologies. This research
project is associated with the RCP Phase XIV which consists in an multidisciplinary study 
between geology, geophysics and petroleum engineering in order to enhance the 
static and dynamic reservoir characterization to maximize the recovery of the field. 
The operator company, Denbury Resources inc. started the CO$_2$ flooding in 2009 and estimate
recover about 60 million barrels that represents 17\% of the original oil in place
(OOIP) of this field that have already produced 54\% after primary and secondary depletion.


\subsection{Delhi Field}

Delhi Field is located in northeastern Louisiana, 40 miles west of 
Loussiana-Mississippi border and 35 miles from Monroe (\ref{fig:location}).
Geologically, lies at the western marging of the Mississippi Interior Salt basin
\cite{ref:alam} and at the southern flank of the Monroe uplift \cite{ref:bloomer} 
(\ref{fig:structure}). The field is 15 miles long by 2--2.5 miles wide covering an
area of 6200 acres. The OOIP is about 357 MMbbl and has an API gravity of 41. 
The field was discovery in 1944, primary and second recovery was about 14\% and 40\%
respectively. The peak production of the field was 18000 barrels per day (BOPD). 
The CO$_2$ continuos flooding program started in november 2009 with the expectation
 of recover an additional 17\%. The first production associated with the injection began
 in march 2010 (\ref{fig:production_new}) and the maximum production since the terciary 
recovery has reach 4,458 barrels BOPD. 

\csmlongfigure[h!]{location}{./chapters/Chapter1_intro/Figures/location}{5in}{caption}

\csmlongfigure[h!]{structure}{./chapters/Chapter1_intro/Figures/structure}{5in}{caption}

\csmlongfigure[h!]{production_new}{./chapters/Chapter1_intro/Figures/production_new}{6in}{caption}


The main reservoir is the Holt Bryant zone that lies at depths between 3,000 and 3,500
feet,  which is composed by Tuscaloosa and Paluxy sandstones, Upper and Lower Cretaceous 
respectively. The Paluxy represents a progradational deltaic depositional environment
 \cite{ref:nick} with average porosity 30\%, permeability of 1000 mD and it range in thickness
from 30 to 75 feet. The Tuscaloosa can be divided into lower and upper units. The lower unit 
consist on transgressive marine depositional processes \cite{ref:doug}, laterelly discontinuos.
The upper Tuscaloosa represent fluvial depositional processes with some interfingering of marine
processes \cite{ref:doug}. The average porosity and permeability of the Tuscaloosa units is 29\%
 and 2700 mD and it range in thickness from 35 to 85 feet.     



For a better understanding in geology of the field the reader is referred 
to \cite{ref:nick} and \cite{ref:doug}.

%\subsection{Data}


%\begin{figure}[[hbtp]
% 	\centering
%  	\includegraphics[width=5.in]{./chapters/Chapter1_intro/Figures/location}
%   	\caption{Location}
%    	\label{location}
%\end{figure}


%(Figure \ref{rcpArea})
%\begin{figure}[h!]
% 	\centering
%  	\includegraphics[width=5.in]{./chapters/Chapter1_intro/Figures/rcpArea}
%   	\caption{RCP area of study}
%    	\label{rcpArea}
%\end{figure}


\subsection{Research Objectives}
Motivation based on what fulanito did ..... I want to continue 

\subsection{Research Objectives}

\begin{itemize}
\item Evaluate the sand connectivity in the Paluxy and Tuscaloosa Formations based on 4D data.  
\item Identification of areas of CO$_2$ saturation and pore pressure changes in the Paluxy and Tuscaloosa Formations. 
\item Integrated interpretation of the results by using production data. 
\item Make recommendations to better sweep the reservoir and prevent well failures.
\end{itemize}



\clearpage


   
   
