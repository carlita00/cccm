\chapter{Introduction}
esto es la introduccion

\subsection{Location}

The Campos Basin is located on the continental shelf of the state of Rio de Janeiro (\ref{fig:campos_basin_location}) in the southeastern part of Brazil between latitudes 20$^o$ and 22$^o$ south and longitudes 40$^o$ and 42$^o$ west. The Campos Basin is the most prolific petroleum basin of Brazil (Winter et al., 2007) (Guardado et al., 1989). The basin comprises an area of about 100,000 Km$^2$. The oilfields in the basin occur in water depths ranging from 80 m to more than 2600 m. The Vitoria High that separates the Campos Basin from the Espirito Santo Basin forms the northern boundary; the Cabo Frio High bounds it to the south, separating it from the Santos Basin. The Campos Basin is open to the east. Sediment fill grades eastward in the form of a sedimentary wedge deposited on the oceanic crust of the South Atlantic. 

	
\begin{figure}[hbtp]
	\begin{center}
	\includegraphics[width=0.9\linewidth]{chapters/Chapter1_intro/Figures/Fig-1_location_Guardado2000}
			\caption[Campos Basin location (Guardado, 2000)]{Campos Basin location (Guardado, 2000). Note: The studied field is not shown due to confidentiality matters.}
			\label{fig:campos_basin_location}
		\end{center}
	\end{figure}	
	
	
	
	
 
\subsection{Campos Basin Geology}

The Campos Basin formed during the break-up of the Gondwana Supercontinent. The break-up involved the splitting of the African and South American plates as a result of doming, rifting, and drifting of the continental crust. It is a typical continental margin basin of the Atlantic type. The Campos Basin presents four main stages of evolution: (1) pre-rift, (2) rift, (3) proto-oceanic gulf or transitional and (4) open marine stages or drift which are shown in \ref{fig:geologic_evolution} (Ojeda,1982) ( Ponte et al, 1980).

\begin{figure}[hbtp]
	\begin{center}
	\includegraphics[width=0.9\linewidth]{chapters/Chapter1_intro/Figures/Fig-2_geo_evol_br_Ponte}
			\caption[Geologic evolution of the Brazilian continental margin (modified from Porto and Dauzacker, from Ponte, 1980)]{Geologic evolution of the Brazilian continental margin (modified from Porto and Dauzacker 1978, from Ponte, 1980). The reservoir studied is this thesis was deposited in the open ocean stage.}
			\label{fig:geologic_evolution}
		\end{center}
	\end{figure}


The pre-rift valley stage occurred from the Late Permian to the Late Jurassic.  The sediments from this stage were not deposited in the Campos Basin because it was located in an interdome position. The rift-valley stage was characterized by the last strong tectonic activity in the Campos Basin. Intense normal faulting, crustal rupture and formation of graben and horst structures associated with an early basic vulcanism took place in this stage. The first marine deposition occurred during Cretaceous time in the proto-oceanic gulf stage. This proto-oceanic gulf stage was tectonically inactive, and marine circulation in the basin was restricted. During this stage a widespread evaporite sequence was deposited that is now the form of salt domes. In the Albian stage, the proto-oceanic gulf developed into an open marine system due to the formation of oceanic crust due to continuous sea floor spreading and subsidence of earlier topographic highs. In this stage a marine carbonate platform was developed just above the evaporites and below the studied reservoir. This was still a tectonically inactive time with subsidence and a relatively low influx of siliciclastics. During the Cenomanian, the basin was tilted eastward due to the subsidence of the ocean floor forming a deep water basin where the deposition of clastic turbidites began to occur (Ponte et al., 1980). 
The geologic evolution of the Campos Basin was controlled by both thermal and isostatic subsidence. Thermal subsidence occurred in response to the cooling and contraction of the lithosphere; isostatic subsidence dominated once thermal subsidence exponentially decreased with time (Ponte et al., 1980).

The sedimentary section of the Campos Basin is subdivided into three megasequences by Guardado (2000) (\ref{fig:stratigraphic_chart}). First, the nonmarine rift megasequence is composed of lacustrine Barremian sedimentary rocks overlying Neocomian basalts (early Neocomian to early Aptian). Second, the transitional evaporitic megasequence was deposited during a period of tectonic quiescence (mid Aptian to early Albian), which represents the beginning of the drift phase. Third, the marine megasequence, which according to Bruhn (1993) (\ref{fig:generalized_section}), can be divided into three megasequences: (1), the carbonate platform megasequence (early to mid Albian), which establishes truly marine conditions in the Campos Basin; (2), the marine transgressive megasequence (late Albian to early Tertiary), in which the turbidity system detailed in this thesis was deposited, and (3), the coastal and deltaic regressive megasequence (early Tertiary to present). 

\begin{figure}[hbtp]
	\begin{center}
	\includegraphics[width=0.9\linewidth]{chapters/Chapter1_intro/Figures/Fig-3_BC_strat_Guardado_2000}
			\caption[Campos Basin stratigraphic chart (Guardado et al., 2000)]{Campos Basin stratigraphic chart (Guardado et al., 2000). The studied field is a deep water turbidite reservoir from the Santonian/Campanian period.}
			\label{fig:stratigraphic_chart}
		\end{center}
	\end{figure}

\begin{figure}[hbtp]
	\begin{center}
	\includegraphics[width=0.9\linewidth]{chapters/Chapter1_intro/Figures/Fig-4_Bruhn}
			\caption[Generalized geological section for the eastern Brazilian marginal basins (Bruhn, 1993).]{Generalized geological section for the eastern Brazilian marginal basins (Bruhn, 1993). The studied reservoir is from the Marine Transgressive Megasequence.}
			\label{fig:generalized_section}
		\end{center}
	\end{figure}


\subsection{Turbidities}

The term turbidite is used to describe a variety of reservoirs, which have been deposited by turbidity currents in relatively deep-water environments (Mutti, 1992). �Turbidity currents are density currents composed of relatively dilute sediment-water mixtures, in which the sediment particles are maintained in suspension primarily by the upward component of the fluid turbulence� (Stow, 1992). In turbidity current, relatively larger sediment particles can be transported over very long distances due to the high density of the fluid turbulence.  A large number of turbidite sandstones have been deposited in submarine fans especially in periods of relative sea-level fall (Mutti, 1992). �Submarine fans are fan-shaped physiographic features composed of terrigenous sediment located seaward of large rivers, deltas and at the end of submarine canyons� (Gary et al., 1974, from Posamentier 1991)   (\ref{fig:submarine_fan}). Sedimentary models of such fan systems typically are subdivided into upper, mid, and lower fan sequences (\ref{fig:model_fan}), each with distinct sand-body geometries, sediment distributions, and lithologic characteristics (Pickering, 1989 and Walker, 1978). These deep water turbidite sandstone are  usually excellent hydrocarbon reservoirs, and they are here in the Campos Basin. A number of papers have been published about turbidites as Chapin et al (1996) and Weimer et al (1994).




\begin{figure}[hbtp]
	\begin{center}
	\includegraphics[width=0.9\linewidth]{chapters/Chapter1_intro/Figures/Fig-5-Fan_Guardado1985}
			\caption[Submarine fan depositional system sketch (Guardado et al., 1986).]{Submarine fan depositional system sketch (Guardado et al., 1986). The studied turbidite reservoir belongs to a partially confined depositional system.}
			\label{fig:submarine_fan}
		\end{center}
	\end{figure}

\begin{figure}[hbtp]
	\begin{center}
	\includegraphics[width=0.9\linewidth]{chapters/Chapter1_intro/Figures/Fig-6_Fan_Walker}
			\caption[Model of submarine-fan deposition (Walker, 1978).]{Model of submarine-fan deposition (Walker, 1978). The studied reservoir can be classified as a mid-fan.}
			\label{fig:model_fan}
		\end{center}
	\end{figure}



\subsection{Field Characteristics}

The field is located in the Campos Basin, Rio de Janeiro, Brazil. It was discovered in 1984, production started in 1985 and water injection started in 2002. It is considered a deep-water oil reservoir. The water depth is from 320m to 780m. The main reservoir consists of  turbidite rocks from the Santonian/Campanian period. The field has been subdivided into an upper zone with great lateral distribution, and a lower zone which is thicker but more restricted in areal extent. The lower zone is associated with an expressive channel. They have good petrophysic characteriztics, with an average porosity of 27 percentage and average permeability of 2500 mD. The oil is 29$^o$ API and 2.1 cP at reservoir conditions. The reservoir has good lateral communication, even though it is cut by several normal faults (Lima et al., 2010).

\subsection{Petroleum System}

The source rock is the lacustrine calcareous black shales of the Lagoa Feia Formation (\ref{fig:geochemical_log}) which were deposited during the Early Cretaceous (Mello, 1988). The source rock has TOC of 2 to 6 wt percent. The hydrogen indices (HI) are as high as 900 mg HC/mg TOC, consistent with a Type I kerogen. Since the area of mature source rock is below the oil reservoir of the field, the oil probably migrated up the faults out of the oil window and into the overlying oil-bearing sandstone (Guardado et al., 2000), namely a vertical migration model. 

\begin{figure}[hbtp]
	\begin{center}
	\includegraphics[width=0.9\linewidth]{chapters/Chapter1_intro/Figures/Fig-7_well_log}
			\caption[Selected geochemical well log from the Campos Basin (Guardado et al., 2000).]{Selected geochemical well log from the Campos Basin showing the hydrocarbon source potential of calcareous shales of the lagoa Feia Formation (Guardado et al., 2000).}
			\label{fig:geochemical_log}
		\end{center}
	\end{figure}



Based on burial history diagrams, the upper part of the Lagoa Feia source rock in the Campos Basin started to generate oil during the Coniacian-Santonian time, reaching peak oil generation during the Miocene, and it is still generating oil today (Solden et al., 1990; from Mello, 1994).

	The hydrocarbon migration pathway model for the oil field involves migration from the Lower Cretaceous source rock to Cretaceous turbidite reservoirs along pre-salt normal faults. The trap includes both structural and stratigraphic components that are related to the evolution of the deep sea fan complex, which deposited the reservoir rock; the deep-water marine shales are the seal rocks. Reconstruction of the burial and thermal history indicate that generation, migration, and accumulation started in the Miocene time and are presently ongoing (Mello et al., 1994). 


\subsection{Seismic Data}
The seismic volumes used in this research were acquired in a ocean bottom cable survey. 
       
\subsubsection{Acquisition}

The 3-D, 4-C ocean-bottom-cable survey (OBS) (\ref{fig:OBS_cartoon}) was conducted by PGS Geophysical at the direction of Petrobras. The 4C means four components, a hydrophone component plus the three orthogonal geophones. Four component OBS surveys have the advantage of being able to also record shear waves, which do not travel through water. The seismic acquisition occurred April to June 2005. The survey was conducted by three vessels: the Ocean Explorer was the recording vessel, the Falcon Explorer was the shooting vessel and the Bergen Surveyor was the cable vessel. The survey consisted of 33 swaths split in three tiers (\ref{fig:pre_plot}). The survey key parameters are summarized in \ref{fig:table_acq} (PGS Geophysical Acquisition Report, 2005). 

\begin{figure}[hbtp]
	\begin{center}
	\includegraphics[width=0.9\linewidth]{./chapters/Chapter1_intro/Figures_ACQ/Fig2_ACQ_cartoon}
			\caption[Acquisition vessel and ocean bottom cables.(PGS Geophysical Acquisition Report, 2005).]{Acquisition vessel and ocean bottom cables.(PGS Geophysical Acquisition Report, 2005).}
			\label{fig:OBS_cartoon}
		\end{center}
	\end{figure}

\begin{figure}[hbtp]
	\begin{center}
	\includegraphics[width=0.9\linewidth]{./chapters/Chapter1_intro/Figures_ACQ/Fig3_ACQ_preplot}
			\caption[Miniature OBS acquisition pre-plot.(PGS Geophysical Acquisition Report, 2005).]{Miniature OBS acquisition pre-plot.(PGS Geophysical Acquisition Report, 2005).}
			\label{fig:pre_plot}
		\end{center}
	\end{figure}
	
	
\begin{figure}[hbtp]
	\begin{center}
	\includegraphics[width=0.9\linewidth]{./chapters/Chapter1_intro/Figures_ACQ/Fig1_ACQ_parameters}
			\caption[Acquisition parameters.(PGS Geophysical Acquisition Report, 2005).]{Acquisition parameters.(PGS Geophysical Acquisition Report, 2005).}
			\label{fig:table_acq}
		\end{center}
	\end{figure}
	



The seismic survey operation was far from straightforward due to the obstacles such as permanently installed production semi-sub and two drilling rigs with extensive anchors in the area (\ref{fig:anchor}). The anchor tension was high and the chain/wires were suspended off the seabed over a 1000 meters radius from each rig. To overcome these problems, various innovative methods were used to minimize the impact on the fold coverage. Given the abundance of installations nearby, there were also vessel noise problems to overcome (PGS Geophysical Acquisition Report, 2005).

\begin{figure}[hbtp]
	\begin{center}
	\includegraphics[width=0.9\linewidth]{./chapters/Chapter1_intro/Figures_ACQ/Fig4_ACQ_anchor}
			\caption[A topographic picture of the receiver depths showing where the cables had to be draped over anchor chains.(PGS Geophysical Acquisition Report, 2005).]{A topographic picture of the receiver depths showing where the cables had to be draped over anchor chains.(PGS Geophysical Acquisition Report, 2005).}
			\label{fig:anchor}
		\end{center}
	\end{figure}


\begin{figure}[h!]
	\begin{center}
	\includegraphics[width=0.9\linewidth]{./chapters/Chapter1_intro/Figures_ACQ/Fig5_ACQ_fold}
			\caption[Example of a fold display (offsets 000-6264 meters) (PGS Geophysical Acquisition Report, 2005).]{Example of a fold display (offsets 000-6264 meters) (PGS Geophysical Acquisition Report, 2005).}
			\label{fig:fold}
		\end{center}
	\end{figure}


\begin{figure}[hbtp]
	\begin{center}
	\includegraphics[width=0.9\linewidth]{./chapters/Chapter1_intro/Figures_ACQ/Fig6_ACQ_shot}
			\caption[Example of a mini common shot gather display.]{Example of a mini common shot gather display, from top to bottom: hydrophone, vertical, inline, and crossline geophones (PGS Geophysical Acquisition Report, 2005). The hydrophone and the vertical geophone resulted in the P-P seismic data, the inline geophone resulted in the P-S data and the crossline geophone were not used due to the weak anisotropy in the area.}
			\label{fig:shot_gather}
		\end{center}
	\end{figure}


 The quality control of the seismic data was done onboard. Brute stack sections were produced for each nominal receiver line, P-P stacks from both the hydrophone and vertical geophone components, and P-S stacks from the inline components. Shot records for each component: hydrophone, vertical geophone, radial geophone and transverse geophone (\ref{fig:shot_gather}), were checked. A fold cube (\ref{fig:fold}) was built up as the survey progressed  (PGS Geophysical Acquisition Report, 2005).


\clearpage

   \subsubsection{Processing}
  
   The 3-D, 4-C OBS survey acquired in 2005 over the target field was processed in 2010 by WesternGeco. In this research I used the P-P, and P-S post-stack data, both volumes were imaged using pre-stack anisotropic time migration. The P-P seismic data was processed using the vertical geophone and the hydrophone components. The P-S seismic data was generated after the inline or radial geophone. 
   
  

 \subsection{Research Objectives}

The main objective in my research is to establish if the 3-D, 4-C OBS seismic data acquired in 2005 and reprocessed in 2010 to charcterize a thin turbidite reservoir in the deep-water Campos basin, Brazil. To achieve this objective, I performed:
\begin{itemize}
\item A seismic interpretation of the P-P and P-S data.
\item A P-P and P-S pos-stack joint inversion generating an acoustic impedance, shear impedance and density volume.
\item Attribute analyses to verify which seismic attributes were best for reservoir characterization.

\end{itemize}

\clearpage


   
   
