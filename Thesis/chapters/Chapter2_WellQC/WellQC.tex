\chapter{Petrophysical properties}


Petrophysical analysis and a rigorous well data quality control (QC)
was done to the logs of wells 140-1 and 159-2 prior to rock physics modeling.
Log analysis is fundamental to understand the petrophysical
properties of the reservoir and for elastic 
parameter estimation.

The analysis started by evaluating the Gamma Ray (GR) log with the cluster
facies analysis done by \cite{ref:doug} which included the integration 
of XRD analysis, mineralogy from point-counting, well logs, core data and seismic data
for depositional environment and facies interpretation. \ref{fig:gr_facies} shows
the GR logs for both wells along with the facies analysis in colors. From this figure, one can 
observe the cleanest facies corresponds to the Tuscaloosa beach/barrier bar, with an
average permeability of 5,200 mD and porosity of 30\%, followed by the Tuscaloosa washover
facies with an average permeability of 3,200 mD and porosity of 32\%. Tuscaloosa fluvial,
and Paluxy facies have an average permeability 700 mD and 1,000 mD respectively. 

Facies modeling showed strong correlation with gamma ray;
therefore, with the lithology. In this chapter, facies
correlation with density, sonic logs and elastic parameters
will be discussed.  


\csmlongfigure[h!]{gr_facies}{./chapters/Chapter2_WellQC/Figures/gr_facies.pdf}{5in}{
Gamma ray logs for wells 159-2 and 140-1. Colors indicate the facies
interpretation done by \cite{ref:doug}. Tuscaloosa beach/barrier bar represents 
the cleanest sandstone while Tuscaloosa fluvial is a shaly sandstone. Black color indicates
the non-reservoir rocks.} 
 

\subsection{Densities and velocities QC}

Caliper log data can indicate washout zones or areas where the borehole is out
of gage. The washout zones may create
anomalous data that need to be corrected for borehole rugosity. Density is one
 of the most affected logs because the measuring tool has to be in contact with 
the wellbore wall \citep{ref:tad}. 
\ref{fig:159-2_QC,fig:140-1_QC} and \ref{fig:140-1_QC} show the caliper and density logs. One can observe there are no washout zones that affecting the density measurements. 
The caliper log is well behaved through the formation, therefore density logs were
 not edited at this stage.


\begin{sidewaysfigure}
\centering
\scalebox{0.5}
{\includegraphics[width=15in]{./chapters/Chapter2_WellQC/Figures/159-2_QC.pdf}}
\caption{Caliper, gamma ray, density and sonic logs of well 159-2. The black color indicates
the observed logs while red color indicates pseudo logs. Gasmmann's equation was used
to generate the density pseudo log. Castagna's equations were used to calculate compressional
and shear velocity.} 
\label{fig:159-2_QC}
\end{sidewaysfigure}

\begin{sidewaysfigure}
\centering
\scalebox{0.5}
{\includegraphics[width=15in]{./chapters/Chapter2_WellQC/Figures/140-1_QC.pdf}}
\caption{Caliper, gamma ray, density and sonic logs of well 140-1. The black color indicates
the observed logs while red color indicates pseudo logs. Gasmmann's equation was used
to generate the density pseudo log. Castagna's equations were used to calculate compressional
and shear velocity.} 
\label{fig:140-1_QC}
\end{sidewaysfigure}


Gardner's equation was used to generate pseudo density logs from compressional velocities
\citep{ref:gardner}:

\begin{equation}
\rho=\alpha V_{p}^{\beta},
  \label{eq:gardner}
\end{equation}

where $\alpha$ is 0.31, $\beta$ is 0.25, $V_{p}$ units are m/s and $\rho$ units are
g/cm$^3$. \ref{fig:159-2_QC} and \ref{fig:140-1_QC} show in red the pseudo density log
and in black the observed log. Discrepancies are observed because Gardner's empirical equation
is more reliable  for well cemented and brine saturated rocks which is not the case 
for Delhi Field rocks. Hence, Gardner's equation is not appropriated for Delhi, calibration of the 
coefficients is recommended in order to have better results, especially in the shale zones. 

Castagna's empirical equations for compressional and shear velocity estimation
were used as a QC tool \citep{ref:castagna}. These equations relate the porosity and the clay 
content for brine saturated rocks:

\begin{equation}
V_{p}=5.81-9.42\phi-2.2C,
  \label{eq:castagna1}
\end{equation}

\begin{equation}
V_{s}=3.89-7.07\phi-2.04C,
  \label{eq:castagna2}
\end{equation}

where $\phi$ is the porosity and $C$ is the clay content. The porosity log used for 
this calculation corresponds to Magnetic-Resonance-Imaging-Log (MRIL). The
tool generates a static magnetic field that magnetizes the spin axes of the protons that 
originally are align with the earth's magnetic field. The tool generates spin echoes
and records the amplitude of the spin echo-train which is proportional to the hydrogen 
protons, therefore to the porosities \citep{ref:george}. Clay content was calculated from
the GR logs.

\ref{fig:159-2_QC} and \ref{fig:140-1_QC} show in red color the pseudo logs and in black the
observed logs. Both wells indicate that predictions have higher frequencies than the
observed values. Lower frequencies show the same behavior in the sandstones, therefore the clay and porosity models
values can be used to model the shear and compressional velocities. However, these modeles overestimate the 
velocities in the shale zones. This behavior is expected for rocks composed primarily of clay \citep{ref:castagna}.  

\subsection{Dry bulk modulus calculation}

One of the most important elastic parameters for rock physics modeling 
is the dry frame bulk modulus of the rock ($K_{dry}$). $K_{dry}$ when calculated by 
using Gassmann's equation is usually noisy, and may contain negative poisson
ratio values \citep{ref:tad}. For this reason, \cite{ref:krief} simple heuristic model
was used to calculate $K_{dry}$:

\begin{equation}
K_{dry}=G \frac{K_{0}}{G_{0}},
  \label{eq:kdry}
\end{equation}

where $K_{0}$ and $G_{0}$ are the bulk and the shear modulus respectively of the mineral matrix. These moduli
were calculated using the Voigt-Reuss-Hill average that is the average between Voigt ($M_{v}$)
and Reuss ($M_{r}$) bounds:

\begin{equation}
M_{H}= \frac{M_{v}+M_{r}}{2}=\frac{1}{2}[\sum_{i=1}^{N}f_{i}M_{i}+(\sum_{i=1}^{N}f_{i}M_{i}^{-1})^{-1})],
  \label{eq:hill}
\end{equation}

the term $M_{i}$ is the moduli and $f_{i}$ is the volume fraction \citep{ref:handbook}. For this modeling,
I assumed the rocks where composed of quartz and clay. This assumption is consistent with the XRD results 
for core 159-2 (see appendix \ref{app:xrds}, \ref{fig:xrd}). \ref{tab:moduli}  shows the bulk and shear modulus values used. 
 
\begin{table}[h!]
\caption{ Bulk and shear modulus used for $K_{dry}$ calculation \label{tab:moduli}}
\begin{center}
\begin{tabular}{ | l | l |l|}
\hline
Mineral      & $K$ (Gpa) & $G$ (Gpa)  \\
\hline
Quartz & 37.0 & 44.0 \\
\hline
Clay & 25.0 &9.0 \\
\hline
\end{tabular}
\end{center}
\end{table}


\ref{fig:159-2QC2} and \ref{fig:140-1QC2} show the calculated $K_{dry}$. The values
range between 3.0 and 5.0 Gpa. The lowest values are observed in the sandstones
with the exception of the Tuscaloosa beach/barrier facies which indicates
 possible cementation.

\csmlongfigure[h!]{159-2QC2}{./chapters/Chapter2_WellQC/Figures/159-2QC2.pdf}{5in}{
Gamma ray, $V_{p}/V_{s}$ ratio of the dry rock frame and $K_{dry}$ of the dry rock frame of well 159-2.}

\csmlongfigure[h!]{140-1QC2}{./chapters/Chapter2_WellQC/Figures/140-1QC2.pdf}{5in}{
Gamma ray, $V_{p}/V_{s}$ ratio of the dry rock frame and $K_{dry}$ of the dry rock frame of well 140-1.}

$V_{p}$/$V_{s}$ dry ratio is another QC shown in \ref{fig:159-2QC2} and \ref{fig:140-1QC2}. One can observe that the 
lowest $V_{p}$/$V_{s}$ ratio is consistent with the cleanest facies and that  
$V_{p}$/$V_{s}$ ratio is a lithology discriminator between sands and shales. Higher 
$V_{p}$/$V_{s}$ ratios indicate high clay content. 

\newpage
\subsection{Fluid substitution}

Rock saturated bulk modulus was calculated using the isotropic form of Gasmmann's equation 
\citep{ref:gasmmann}:
\begin{equation}
K_{sat}= K_{dry}+\frac{(1-K_{dry}/K_{0})^{2}}{\phi/K_{fl}+(1-\phi)/K_{0}-K_{dry}/K_{0}^2},
  \label{eq:gasmmann}
\end{equation}

where $K_{dry}$ is the dry bulk frame modulus calculated in Equation \ref{eq:kdry}, $K_{0}$ 
is the bulk modulus of the mineral matrix, $\phi$ is the total porosity and $K_{fl}$ is 
the fluid bulk modulus calculated by using Hill's average (\ref{eq:hill}). The oil saturation
values were obtain by the MRIL logs. The reader is referred to \cite{ref:george} for
a review of nuclear magnetic logs (NMR).

The fluid properties used for the modeling are shown in \ref{tab:fluid}. The moduli
of the fluid were calculated using Batzle and Wang spreadsheet and are shown in \ref{tab:flag}. 


\begin{table}[h!]
\caption{Fluid properties used for $K_{sat}$ modeling \label{tab:fluid}}
\begin{center}
\begin{tabular}{ | l | l |}
\hline
Temperature      & 137 F  \\
\hline
Pressure      & 1450 PSI  \\
\hline
Oil Gravity & 43.9 API \\
\hline
Gas Gravity & 0.65 \\
\hline
Salinity & 80,000 \\
\hline
\end{tabular}
\end{center}
\end{table}


\begin{table}[h!]
\caption{Brine and oil moduli \label{tab:flag}}
\begin{center}
\begin{tabular}{ | l | l |l|}
\hline
Fluid type & $\rho$ ($g/cm^3$)& K (GPa) \\
\hline
Brine      &  1.03 & 2.81  \\
\hline
Oil      & 0.69 & 0.82   \\
\hline
\end{tabular}
\end{center}
\end{table}

$V_{p}$ was calculated as another QC tool by using:
 
\begin{equation}
V_{p}=\sqrt{\frac{K_{sat}+4/3 G_{sat}}{\rho_{sat}}}, 
  \label{eq:gassmann}
\end{equation}

where $K_{sat}$ is the bulk modulus of the rock calculated in \ref{eq:gasmmann}, $G_{sat}$ is the shear
modulus calculated from the logs and $\rho_{sat}$ is the rock density.


\ref{fig:159-2_vp} and \ref{fig:140-1_vp} show the oil saturation log, the $K_{sat}$, 
the $V_{p}$ observed and the $V_{p}$ calculated log. Well 140-1 in \ref{fig:140-1_vp}  shows good correlation
 between the observed and calculated $V_{p}$ which indicates consistency between the 
$K_{dry}$, oil saturation log and the recorded sonic logs. Well 159-2 in  \ref{fig:159-2_vp} 
does not show good correlation between the observed and calculated $V_{p}$. Calculated $V_{p}$ 
is underestimating the sonic log values, this behavior is possibly related to invasion effects of 
the water-based mud which could be penetrating the formation with brine. The invasion effect
can be seen in \ref{fig:invasion} where the shallow and deep resistivity logs show discrepancies
in the oil saturated zones.

\cite{ref:han} suggest that velocities and densities affected by invasion are difficult to correct.
Incorrect values can yield wrong reflectivity models producing misties with the seismic. 
I use the velocity and density pseudo-logs generated in this chapter for 
the rock physics modeling and the seismic ties. 


\csmlongfigure[h!]{159-2_vp}{./chapters/Chapter2_WellQC/Figures/159-2_vp.pdf}{4.5in}{
Gamma ray, oil saturation, and $V_{p}$ of well 159-2. Black color indicates the recorded sonic. 
Red color indicates the pseudo-log.}

\csmlongfigure[h!]{140-1_vp}{./chapters/Chapter2_WellQC/Figures/140-1_vp.pdf}{4.5in}{
Gamma ray, oil saturation, and $V_{p}$ of well 159-2. Black color indicates the recorded sonic. 
Red color indicates the pseudo-log.}

\csmlongfigure[h!]{invasion}{./chapters/Chapter2_WellQC/Figures/invasion.pdf}{4.5in}{10 inch and 90
inch resistivity logs of well 159-2. Note how the resistivity values changes in the 
oil saturated zones due to invasion effects.}

\subsection{QC Crossplot}

$V_{p}/V_{s}$ ratio vs acoustic impedance is a crossplot commonly used in rock physics
because we can evaluate well-log data and seismic inversion data \citep{ref:avseth}.
\ref{fig:ip_vpvs} shows the crossplot, the color scale is the clay volume obtain
from GR logs. A clear separation between sandstones and shales is observed.
This qualitative result is encouraging for the rock physics modeling and 
joint inversion framework. The next chapter will provide a calibration of 
rock physics models based on logs with the objective to do a quantitave interpretation
of the data.


\csmlongfigure[h!]{ip_vpvs}{./chapters/Chapter2_WellQC/Figures/ip_vpvs.pdf}{6in}{
$V_{p}/V_{s}$ ratio vs acoustic impedance for wells 140-1 and 159-2}


\newpage\mbox{}\newpage
