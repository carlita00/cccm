\chapter{Inversion}

A key benefit of having both P-P and P-S data is to use them together to more tightly constrain the inversion process which provides reservoir layer properties.
The objective of the inversion was to  transform an interface property into a layer property.  According to Tarantola (2005), �the inverse problem consists of using the actual result of some measurements to infer the values of the parameters that characterize the system�. In this case, the measurements were the P-P and P-S post stack data and the well logs. The parameters that characterize the system are the acoustic impedance, the shear impedance and the density volume. The studied reservoir is a layer of sandstone filled with oil, the sandstone has high porosity and a lower impedance than the surrounding shales. To generate acoustic impedance, shear impedance and density volume, I performed a P-P and P-S Joint Inversion (Hampson and Russell, 2008b) (\ref{fig:inversion_flow}).  To perform this inversion, I conducted a well tie to create P-P and P-S time-depth curves. Then I converted the P-S data from P-S time to P-P time domain. The second step was to build a joint inversion low-frequency model. The model was created using the well logs and seismic horizons. After some tests, I use only one well in the model creation, the Well 01D, the only well that penetrates the reservoir and has P-wave velocity and S-wave velocity. The low-frequency model created for the joint inversion is done with the density, S-wave and P-wave logs. The input to the inversion were the  P-P and P-S wavelets extracted in P-P time, the P-P and P-S seismic volumes and the joint inversion low-frequency model. The main outputs were the acoustic impedance (Zp), the shear impedance (Zs) and the density volume.






\begin{figure}[hbtp]
	\begin{center}
	\includegraphics[width=0.9\linewidth]{./chapters/Chapter3_Inversion_latex/Figures/Fig1_inversion_procedures}
			\caption[P-P and P-S joint inversion procedures  (Hampson and Russell (2008b).]{P-P and P-S joint inversion procedures (Hampson and Russell (2008b). The input data were the P-P and P-S post stack data, the wavelets in P-P time extracted from the P-P and the P-S data, and the low frequency model build after the well logs (P-wave velocity, S-wave velocity and density) and seismic horizons. The main outputs were the acoustic impedance (Zp), the shear impedance (Zs) and the density volume.}
			\label{fig:inversion_flow}
		\end{center}
	\end{figure}


\subsection{Model Building}

The model based P-P, P-S joint inversion requires a low frequency model to run, but the seismic data does not have low frequency. The generation of the low-frequency model is the most important step of the inversion process. In order to achieve reasonable outputs from the inversion process, I tried a range of values for several tests to create the the low-frequency model.
	
	Test 1: I did the horizon matching between the top of the carbonate in the P-P and in the P-S seismic volumes. The top of the carbonate map has some uncertainty, and the pick was difficult to make in certain locations. This uncertainty in the top of the carbonate interpretation, in both P-P and P-S seismic volumes, created some distortions in the P-P and P-S velocity model. Consequently, the results from the inversion using this model has all the errors associated with the P-P and P-S top of carbonate interpretations problems.
	
	Test 2: As the horizon match is not a mandatory step in the inversion process, I skipped it. The results were good near the well. Away from the well I did not have a way to match the P-P and P-S data and the impedance results were less reliable.
	
	Test 3: I used just one horizon to match the P-P and P-S seismic data. The horizon used was the top of the pebbly zone. In the model creation, I used both top of the pebbly zone and the top of the carbonate platform. The wells used were the ones with higher synthetic and seismic correlation coefficient for both P-P and PS data. I used these three wells for the low frequency model generation.
	
	Test 4: I used only one horizon to match the P-P and P-S seismic data. The horizon used was the top of the reservoir. In the model creation, I used the top of the pebbly zone, the top of the reservoir and the top of the carbonate. I used only one well (Well 01D) in the model, I chose it because this is the only well that has P-wave and S-wave logs that penetrate the oil reservoir. The result of this modeling is a model of Vp, Vs and density (\ref{fig:density_model}).

	A joint inversion was done using each one of the low-frequency model tested. Based on the results of the inversion, the Test 4 was interpreted to be the preferred model.

\begin{figure}[hbtp]
	\begin{center}
	\includegraphics[width=0.9\linewidth]{./chapters/Chapter3_Inversion_latex/Figures/Fig2_Model_July27_w01D_DEN_Inline1334}
			\caption[Crossline 591, density model.]{Crossline 591, density model. This model was created from the well logs and the main seismic horizons.}
			\label{fig:density_model}
		\end{center}
	\end{figure}



\subsection{P-P and P-S well tie}

I used the well log markers interpreted from Petrobras geologists as a guide to tie the well and seismic. Most of the well logs have the pebbly zone interpreted. I tied the top of the pebbly zone seismic horizon interpreted in time, with the pebbly zone well marker in the log interpreted in depth. The selection of the three wells was based mainly on the presence of P-wave and S-wave velocity logs, and secondarily on a good correlation with the seismic and well. The wells selected were 01D (the only one with an S-wave velocity log in the reservoir) and the wells 03 and 04 (vertical wells with good correlation coefficient   between the seismic and the logs). 

The P-P and P-S well ties require a wavelet which can be extracted from the data. I chose, however to use Ricker wavelets in the synthetic creation as they were very similar to the extracted wavelets. Different frequencies were tested to see which one was most similar to the seismic data in the reservoir level (Figures 3.4, 3.5, 3.6, 3.7 and 3.8). 
	Well 01D was a key well to my inversion. At Well 01D, the P-P synthetic seismic has a correlation coefficient of 81 percent with the P-P seismic data in a window between 2600 ms and 3075 ms (\ref{fig:synth_01D_PP}). The wavelet used was a 20 Hz Ricker   in the P-P synthetic and a 13 Hz Ricker on the P-S synthetics. The P-S synthetic has a correlation coefficient of 65 percent with the P-S seismic data in a window between 4550 ms and 5250 ms (\ref{fig:synth_01D_PS}).  

\begin{figure}[hbtp]
	\begin{center}
	\includegraphics[width=0.9\linewidth]{./chapters/Chapter3_Inversion_latex/Figures/Fig3_synth_01_PP}
			\caption[P-P synthetic-field data match at well 01D. ]{P-P synthetic-field data match at well 01D. In blue is the synthetic seismic generated from the P-wave and density logs. In red the trace from the seismic extracted around the well and repeated five times. In black are the nine traces nearest to the well. Note the similarities between the synthetic and the traces around the well; the correlation coefficient is 81 percent. The wavelet used was a 20 Hz Ricker  wavelet.}
			\label{fig:synth_01D_PP}
		\end{center}
	\end{figure}



\begin{figure}[hbtp]
	\begin{center}
	\includegraphics[width=0.9\linewidth]{./chapters/Chapter3_Inversion_latex/Figures/Fig4_synth_01_PS}
			\caption[P-S synthetic-field data match at well 01D.]{P-S Synthetic-field data match at well 01D. In blue is the synthetic seismic generated from the S-wave and density logs. In red is the trace from the seismic extracted around the well and repeated five times, and in black are the nine traces from the seismic data nearest to the well. Note the similarities between the synthetic and the traces around the well; the correlation coefficient is 65 percent. The wavelet used was a 13 Hz Ricker wavelet. }
			\label{fig:synth_01D_PS}
		\end{center}
	\end{figure}



 The P-P synthetic from well 03 (the wavelet used was a 20 Hz Ricker) has a correlation coefficient of 71 percent with the P-P seismic data in a window between 2500 ms and 2800 ms (\ref{fig:synth_03_PP}). The P-S synthetic has a correlation coefficient of 83 percent with the P-S seismic data in a window between 4400 ms and 4950 ms (\ref{fig:synth_03_PS})  with a 10 Hz Ricker wavelet.  


\begin{figure}[hbtp]
	\begin{center}
	\includegraphics[width=0.9\linewidth]{./chapters/Chapter3_Inversion_latex/Figures/Fig5_synth_03_PP}
			\caption[P-P synthetic-field data match at well 03. ]{P-P synthetic-field data match at well 03. In blue is the synthetic seismic generated from the P-wave and density logs. In red is the trace from the seismic extracted nearest the well and repeated five times, and in black are the nine traces from the seismic data nearest to the well. Note the similarities between the synthetic and the traces around the well; the correlation coefficient is 71 percent. The wavelet used was a 20 Hz Ricker wavelet.}
			\label{fig:synth_03_PP}
		\end{center}
	\end{figure}
	
	
\begin{figure}[hbtp]
	\begin{center}
	\includegraphics[width=0.9\linewidth]{./chapters/Chapter3_Inversion_latex/Figures/Fig6_synth_03_PS}
			\caption[P-S synthetic-field data match at well 03. ]{P-S synthetic-field data match at well 03. In blue is the synthetic seismic generated from the S-wave and density logs. In red is the trace from the seismic extracted nearest the well and repeated five times, and in black are the nine traces from the seismic data nearest to the well. Note the similarities between the synthetic and the traces around the well; the correlation coefficient is 76 percent. The wavelet used was a 10 Hz Ricker.}
			\label{fig:synth_03_PS}
		\end{center}
	\end{figure}	


The third well was well 04.  The P-P synthetic has a correlation coefficient of 83 percent with the P-P seismic data in a window between 2500 ms and 2950 ms (\ref{fig:synth_04_PP}). In the P-P synthetic the wavelet used was a 15 Hz Ricker.  In the P-S synthetic the wavelet used was a 10 Hz Ricker. The P-S synthetic has a correlation coefficient of 80 percent with the P-S seismic data in a window between 4400 ms and 5100 ms (\ref{fig:synth_04_PS}) .  


\begin{figure}[hbtp]
	\begin{center}
	\includegraphics[width=0.9\linewidth]{./chapters/Chapter3_Inversion_latex/Figures/Fig7_synth_04_PP}
			\caption[P-P synthetic-field data match at well 04. ]{P-P synthetic-field data match at well 04. In blue is the synthetic seismic generated from the S-wave and density logs. In red is the trace from the seismic extracted nearest to the well, and in black are the nine traces from the seismic data nearest to the well. Note the similarities between the synthetic and the traces around the well; the correlation coefficient is 83 percent. The wavelet used was a 15 Hz Ricker.}
			\label{fig:synth_04_PP}
		\end{center}
	\end{figure}

\begin{figure}[hbtp]
	\begin{center}
	\includegraphics[width=0.9\linewidth]{./chapters/Chapter3_Inversion_latex/Figures/Fig8_synth_04_PS}
			\caption[P-S synthetic-field data match at well 04. ]{P-S synthetic-field data match at well 04. In blue is the synthetic seismic generated from the S-wave and density logs. In red is the trace from the seismic extracted nearest to the well, and in black are the nine traces from the seismic data nearest to the well. Note the similarities between the synthetic and the traces around the well; the correlation coefficient is 80 percent. The wavelet used was a 10 Hz Ricker.}
			\label{fig:synth_04_PS}
		\end{center}
	\end{figure}



\subsection{P-P and P-S wavelets}
I extracted a statistical P-P wavelet from the P-P seismic data (\ref{fig:PP_wavelet}) and a statistical P-S wavelet from the P-S seismic data in the P-P time domain (\ref{fig:PS_wavelet}) to be used as input for the P-P and P-S post-stack joint inversion (Hampson and Russell (2008b)).

\begin{figure}[hbtp]
	\begin{center}
	\includegraphics[width=0.9\linewidth]{./chapters/Chapter3_Inversion_latex/Figures/Fig9_wavelet_PP}
			\caption[P-P wavelet extracted from the P-P seismic data in a time window between 2300 ms and 3200 ms.]{P-P wavelet extracted from the P-P seismic data in a time window between 2300 ms and 3200 ms. In the left, the wavelet in time and in the right the wavelet amplitude spectrum.}
			\label{fig:PP_wavelet}
		\end{center}
	\end{figure}

\begin{figure}[hbtp]
	\begin{center}
	\includegraphics[width=0.9\linewidth]{./chapters/Chapter3_Inversion_latex/Figures/Fig10_wavelet_PS}
			\caption[P-S wavelet extracted from the P-S seismic data in a P-P time domain.]{P-S wavelet extracted from the P-S seismic data in a P-P time domain. The time window is between 2300 ms and 3200 ms. In the left the wavelet in time and in the right the amplitude spectrum.}
			\label{fig:PS_wavelet}
		\end{center}
	\end{figure}


\subsection{Inversion Parameters}
The joint inversion P-P time range was between 2000ms to 3500ms.\\
The input data for this inversion were:\\
	-	the P-P and P-S migrated post-stack volumes\\
	-	Vp, Vs and density models\\
	-	P-wave and S-wave logs\\
	-	Regression coefficients that define the relationships between Zp and density, and between Zp and Zs.\\

The regression coefficients control the background relationship for the inversion parameters and are determined from the cross plots (\ref{fig:crossplots}) of the natural logarithms of the impedances Zp and Zs and the density (as derived from the well data). The coefficients are based on these equations, which are used to reduce the non-uniqueness of the inversion:

\begin{equation}
ln(Z_S)=k \ ln(Z_P) + k_c +  \Delta L_S 
\end{equation}

\begin{equation}
ln(\rho)=m \ ln(Z_P) + m_c +  \Delta L_{\rho}
\end{equation}

where,
k = angular coefficient of the linear curve from the cross-plot between Zs and Zp.\\
m = angular coefficient of the linear curve from the cross-plot between density and Zp.\\
L = distance of the points in the cross-plots from the linear curve that best fit the data.\\


\begin{figure}[hbtp]
	\begin{center}
	\includegraphics[width=0.9\linewidth]{./chapters/Chapter3_Inversion_latex/Figures/Fig12_Zs_Zp_crossplot_JointInversionParameters_July27}
			\caption[Cross-plots  of the natural logarithms of the impedances Zp and Zs and the density (as derived from the well 01D).]{Cross-plots  of the natural logarithms of the impedances Zp and Zs and the density (as derived from the well 01D). }
			\label{fig:crossplots}
		\end{center}
	\end{figure}


\subsection{Inversion Results}


The P-P and P-S post-stack joint inversion assisted the reservoir characterization. The shear-impedance attribute map (\ref{fig:map_Zs}) extracted in a 30 ms window below the top of the reservoir has better resolution than the P-S amplitude seismic data and the inverted density improved the image in the reservoir lower zone. The minimum acoustic impedance attribute map (\ref{fig:map_Zp}), extracted in a 30 ms window below the top of the reservoir is similar to the minimum amplitude or most negative amplitude map extracted in  a 10 ms centered window at the top of the reservoir from the P-P seismic data. The size of attribute extraction windows are different because the extraction of an interface property requires a window size proportional to the size of the seismic resolution and to extract a layer property I need a window the size proportional to the size of the layer. 

The inversion results can be thought of as the transformation of amplitudes, caused by interface properties into a layer properties calibrated with the well logs. An important advantage of the layer properties from the inversion is that it has values that can be correlated with the well logs and used to populate a geologic model. 


The acoustic impedance (\ref{fig:Inline_1334_Zp}), the shear impedance (\ref{fig:Inline_1334_Zs}), and the density volume (\ref{fig:Inline_1334_Den}) are consistent with the well logs. The reservoir has lower acoustic impedance, lower shear impedance and lower density than the surrounding shale.  


\begin{figure}[hbtp]
	\begin{center}
	\includegraphics[width=0.9\linewidth]{./chapters/Chapter3_Inversion_latex/Figures/Fig14_map_Simpedance_min25msbelow}
			\caption[Inverted shear-impedance attribute map, extracted in a 25 ms window below the top of the reservoir.]{Inverted shear impedance attribute map, extracted in a 25 ms window below the top of the reservoir.}
			\label{fig:map_Zs}
		\end{center}
	\end{figure}



\begin{figure}[hbtp]
	\begin{center}
	\includegraphics[width=0.9\linewidth]{./chapters/Chapter3_Inversion_latex/Figures/Fig13_map_Pimpedance_min25msbelow}
			\caption[Inverted acoustic-impedance attribute map, extracted in a 25 ms window below the top of the reservoir.]{Inverted acoustic impedance attribute map, extracted in a 25 ms window below the top of the reservoir.}
			\label{fig:map_Zp}
		\end{center}
	\end{figure}
  


\begin{figure}[hbtp]
	\begin{center}
	\includegraphics[width=1.0\linewidth]{./chapters/Chapter3_Inversion_latex/Figures/Fig15_inline1334_Pimpedance_zoomwell01D}
			\caption[ Acoustic impedance, inline 1358. ]{ Acoustic impedance, inline 1358 with acoustic impedance well log from well 11.}
			\label{fig:Inline_1334_Zp}
		\end{center}
	\end{figure}

\begin{figure}[hbtp]
	\begin{center}
	\includegraphics[width=1.0\linewidth]{./chapters/Chapter3_Inversion_latex/Figures/Fig16_inline1334_Simpedance_zoomwell01D}
			\caption[ Shear impedance, inline 1330.]{ Shear impedance, inline 1330 with shear impedance and density well log from well 01D.}
			\label{fig:Inline_1334_Zs}
		\end{center}
	\end{figure}


\begin{figure}[hbtp]
	\begin{center}
	\includegraphics[width=1.0\linewidth]{./chapters/Chapter3_Inversion_latex/Figures/Fig16_density}
			\caption[ Density , crossline 399.]{ Density, crossline 399 with density well log from well 05D.}
			\label{fig:Inline_1334_Den}
		\end{center}
	\end{figure}
	

\subsection{Fluid replacement modeling}

Fluid substitution modeling was performed using the Gassmann's equation. I studied the contrast in the shear impedance and in the acoustic impedance between the reservoir and the overlying shale. The well logs used were the P-wave velocity , the S-wave velocity and the density. The well used was the well 01D. I calculated the P-wave and the S-wave velocities and the density in the reservoir were calculeted using three scenarios: (1) 80 percent filled with oil, 20 percent with brine, (2) 80 percent filled with brine, 20 percent filled with oil, and (3) 80 percent filled with gas, 20 percent filled with brine (\ref{fig:FRM_res}). In the overlying shale  values were held constant with the shale saturated with 100 percent with brine (\ref{fig:FRM_shale}). The fluid and reservoir properties used in the modeling are shown in Table 3.1. Using the three fluid replacement scenarious, the acoustic and shear impedance variations were calculated between the shale and the reservoir. The results are shown in \ref{fig:FRM_table2}:


\begin{figure}[hbtp]
	\begin{center}
	\includegraphics[width=1.0\linewidth]{./chapters/Chapter3_Inversion_latex/Figures/FRM_res}
			\caption[Fluid replacement modeling in the reservoir.]{Fluid replacement modeling in the reservoir.}
			\label{fig:FRM_res}
		\end{center}
	\end{figure}
	
\begin{figure}[hbtp]
	\begin{center}
	\includegraphics[width=1.0\linewidth]{./chapters/Chapter3_Inversion_latex/Figures/FRM_shale}
			\caption[Brine saturated overlying shale, density, S-wave and P-wave velocity values.]{Brine saturated overlying shale, density, S-wave and P-wave velocity values.}
			\label{fig:FRM_shale}
		\end{center}
	\end{figure}


\begin{table} \label{table_fluid_properties} 
\begin{center} 
\begin{tabular}{| c | c | } 
\hline 
Temperature                & 89 $^o$ degree  Celsius     \\ 
\hline 
Oil gravity         &  29 API       \\ 
\hline 
Gas gravity           & 0.79        \\ 
\hline 
Gas-oil-ration   &  421 scft/sbbl       \\ 
\hline
Salinity   &  100,000 ppm  \\
\hline
Pressure   &    3982 psi  \\
\hline 
\end{tabular} 
\caption{Fluid properties used in the fluid replacement modeling.} 
\end{center} 
\end{table}





\begin{figure}[hbtp]
	\begin{center}
	\includegraphics[width=0.9\linewidth]{./chapters/Chapter3_Inversion_latex/Figures/FRM_table2}
			\caption[Fluid replacement modeling results table.]{Fluid replacement modeling results table.}
			\label{fig:FRM_table2}
		\end{center}
	\end{figure}

 The results from the fluid replacement model matches the seismic results. The P-P amplitude shows a strong trough in the top of the oil reservoir indicating that the acoustic impedance contrast when the reservoir is oil saturated  is strong, the acoustic impedance is very sensible to the fluid saturation. The P-S amplitude has a weak trough in the top of the reservoir indicating that the shear velocity is not sensible to fluid content. In the inversion results the reservoir has a lower acoustic impedance and a lower shear impedance than the overlying shales matching the results from the modeling done above.




 

