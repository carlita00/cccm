\chapter{Dynamic rock physics modeling}

The geological processes that affects siliciclastic sediments and rock porosities have been 
studied by many authors (references). Trends in velocity 
porosity relations can indicate diagenetic or depositional processes that 
give information about the rock fabric \citep{ref:avseth}. \ref{fig:processes} shows 
a summary of the processes that controls porosity on the clastic rocks. Steep slopes
indicates stiffness, compaction and cementation and flatter slopes corresponds
to sorting and clay content. Therefore, evaluating the data in the rock physics framework
helps to understand the reservoir nature.
 
\csmlongfigure[h!]{processes}{./chapters/Chapter3_Rock_physics/Figures/processes}{4.0in}{
Depositional and diagenetic processes that affects sandstones porosity. Taken from
\cite{ref:avseth} }

In this study I use rock physics modeling as a tool to integrate  core, petrophysical, 
XRD, facies data and 4D joint seismic inversion to make a quantitative
interpretation of Delhi Field reservoir changes due to CO$_2$ injection.
For the static rock physics modeling I follow the workflow suggested by \cite{ref:avseth}. This 
consist on a hybrid  approach that uses theoretic, heuristic and empirical
models built for high porosity sandstone reservoirs. The used models are best known as friable
sand, contact cement and constant cement models. 

For the dynamic rock physics modeling I follow the workflow suggested in \cite{ref:dynamic} that
relates relative changes in $V_{p}/V{s}$ ratio and acoustic impedance to pore pressure
and fluid changes. Then, I extend this qualitative analysis into quantitative calculating
pore pressure maps for each facies. The pore pressure calculation is important property for maximize
hydrocarbon recovery, for reservoir management and for drilling safety.

The detail workflow of the modeling is shown in \ref{fig:workflow}.
The subsection of the chapters corresponds to the colors of the workflow

\ref{fig:workflow}
\csmlongfigure[h!]{workflow}{./chapters/Chapter3_Rock_physics/Figures/workflow2}{5.0in}{logs}


\subsection{Rock physics models}
bla

\subsection{Model calibration}
bla
bla \ref{fig:theoryModel}
\csmlongfigure[h!]{theoryModel}{./chapters/Chapter3_Rock_physics/Figures/theoryModel}{5in}{logs}


bla \ref{fig:model_calibration}
\csmlongfigure[h!]{model_calibration}{./chapters/Chapter3_Rock_physics/Figures/model_calibration}{5in}{logs}

bla \ref{fig:model_faciesQ85}
\csmlongfigure[h!]{model_faciesQ85}{./chapters/Chapter3_Rock_physics/Figures/model_faciesQ85}{7in}{logs}

bla \ref{fig:thinTW}
\csmlongfigure[h!]{thinTW}{./chapters/Chapter3_Rock_physics/Figures/thinTW}{5in}{logs}

bla \ref{fig:facies_modelingP}
\csmlongfigure[h!]{facies_modelingP}{./chapters/Chapter3_Rock_physics/Figures/facies_modelingP}{5in}{logs}

bla \ref{fig:thinTB}
\csmlongfigure[h!]{thinTB}{./chapters/Chapter3_Rock_physics/Figures/thinTB}{5in}{logs}

bla \ref{fig:facies_modelingTB}
\csmlongfigure[h!]{facies_modelingTB}{./chapters/Chapter3_Rock_physics/Figures/facies_modelingTB}{5in}{logs}

bla \ref{fig:thinTF}
\csmlongfigure[h!]{thinTF}{./chapters/Chapter3_Rock_physics/Figures/thinTF}{5in}{logs}

bla \ref{fig:facies_modelingTF}
\csmlongfigure[h!]{facies_modelingTF}{./chapters/Chapter3_Rock_physics/Figures/facies_modelingTF}{5in}{logs}

\newpage
\subsection{Static Rock Physics template}

bla \ref{fig:staticModelP}
\csmlongfigure[h!]{staticModelP}{./chapters/Chapter3_Rock_physics/Figures/staticModelP}{7in}{logs}



\ref{tab:modeling}
\begin{table}[h]
\caption{ Modeling parameters \label{tab:modeling}}
\begin{center}
\begin{tabular}{ | l | l | l|  l | l | l|}
\hline
  Facies            &  n & Quartz & Clay & Siderite & P$eff$ (Mpa)  \\
\hline
Tusc Fluvial        &                 8.64 &   0.7  &  0.2 &     0.1  &  12.5  \\
\hline
Tusc Beach/Barrier  & 12.96                &   0.97  &  0.03 &     0.0  &  12.5  \\
\hline
Tusc Washover/Transitional and Paluxy  & 10.36    &   0.85  &  0.15 &     0.0  &  12.5  \\
\hline
\end{tabular}
\end{center}
\end{table}



%\subsubsection{Static rock physics template}
%\subsubsection{Dynamic rock physics template}
