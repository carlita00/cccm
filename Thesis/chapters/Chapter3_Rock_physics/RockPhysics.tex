\chapter{Dynamic rock physics modeling}

In this study I use rock physics modeling as a tool to integrate  core, petrophysical, 
XRD, facies data and 4D joint seismic inversion to make a quantitative
interpretation of Delhi Field reservoir changes due to CO$_2$ injection.

The geological processes that affects siliciclastic sediments and rock porosities have been 
studied by many authors (references). Trends in velocity 
porosity relations can indicate diagenetic or depositional processes that 
give information about the rock fabric \citep{ref:avseth}. \ref{fig:processes} shows 
a summary of the processes that controls porosity on the clastic rocks. Steep slopes
indicates stiffness, compaction and cementation and flatter slopes corresponds
to sorting and clay content. Therefore, evaluating the data in the rock physics framework
helps to understand the reservoir nature.
 
\csmlongfigure[h!]{processes}{./chapters/Chapter3_Rock_physics/Figures/processes}{4.0in}{
Depositional and diagenetic processes that affects sandstones porosity. Taken from
\cite{ref:avseth} }

For the static rock physics modeling I follow the workflow suggested by \cite{ref:avseth}. This 
consist on a hybrid  approach that uses theoretic, heuristic and empirical
models built for high porosity sandstone reservoirs. The used models are best known as friable
sand and cemented sand models. 

For the dynamic rock physics modeling I follow the workflow suggested in \cite{ref:dynamic} that
relates relative changes in $V_{p}/V{s}$ ratio and acoustic impedance to pore pressure
and fluid changes. Then, I extend this qualitative analysis into quantitative calculating
pore pressure maps for each facies. The pore pressure calculation is important property for maximize
hydrocarbon recovery, reservoir management and drilling safety.

The work of this chapter can be summarize in the workflow shown in \ref{fig:workflow}.
The are four main processes highlighted in colors which corresponds to subsections. Yellow,
green, blue and purple corresponds to subsections: rock physics models, model calibration,
static modeling and dynamic modeling, each subsection will be explain in detail.

\csmlongfigure[h!]{workflow}{./chapters/Chapter3_Rock_physics/Figures/workflow}{5.3in}{
Rock physics modeling workflow. Colors: yellow, green, blue and purple indicates the subsections
rock models, model calibration, static modeling and dynamic modeling respectively.}


\subsection{Rock physics models}

Friable and cemented sands models are based in granular media. Granular media provide
insights about porosity, sorting, cementation and stress effects of sandstones by studying
packing of spheres that ideally represents rock grains \citep{ref:handbook}.    
Although none of these models represents the real rocks they are very powerful to
link reservoir and elastic properties allowing to predict the seismic signature
in the reservoir when carefully applied. 

%As \cite{ref:draper} suggested ``All models are wrong, but some are useful".

\subsubsection{Friable sand model}
This model was developed by \cite{ref:dvorkin}, consists in finding the elastics moduli 
of the dry rock when no there is no cement in the grain contacts. To describe the end member
Hertz Mindlin theory is used to mimic well sorted sands where the effective moduli is giving by: 

\begin{equation}
K_{HM}=\left[\frac{n^2(1-\phi_{C})^2G_{s}^2}{18\pi^2(1-\upsilon_{s})^2}P\right]^{1/3},
  \label{eq:KHM}
\end{equation}

\begin{equation}
G_{HM}=\frac{5-4\upsilon_{s}}{5(2-\upsilon_{s})}\left[\frac{3n^2(1-\phi_{C})^2G_{s}^2}{2\pi^2(1-\upsilon_{s})^2}P\right]^{1/3},
  \label{eq:GHM}
\end{equation}

where $\phi_{C}$ is the critical porosity (minimum porosity at which the rock is fluid supported),
$\upsilon_{s}$ and $G_{s}$ are the poisson ratio and the shear modulus of the mineral frame respectively,
 P is the effective pressure and n is the coordination number (number of contacts per grain).

The effective pressure is corresponds to the difference between overburden and pore pressure: 
\begin{equation}
P_{eff}=P_{c}-P_{p},
  \label{eq:P}
\end{equation}

the term $P_{c}$ is the overburden pressure assumed to be 1 psi/ft \citep{ref:sidra}, the pore pressure 
was modeled to range between from 5.0 to 19.0 MPa.

The modified Hashin-Shritkman (HS) lower bound, a model that connects the end member moduli at $\phi_{C}$ with 
the dry rock mineral frame at zero porosity is used to find the elastic moduli for different porosities,

\begin{equation}
K_{dry}=\left[\frac{\frac{\phi}{\phi_{C}}}{K_{HM}+\frac{4}{3}G_{HM}}+\frac{1-\frac{\phi}{\phi_{C}}}{K+\frac{4}{3}G_{HM}}\right]^{-1}-\frac{4}{3}G_{HM},
  \label{eq:kdry}
\end{equation}

\begin{equation}
G_{dry}=\left[\frac{\frac{\phi}{\phi_{C}}}{G_{HM}+\frac{G_{HM}}{6}\left(\frac{9K_{HM}+8G_{HM}}{K_{HM}+2G_{HM}}\right)}+\frac{1-\frac{\phi}{\phi_{C}}}{G+\frac{G_{HM}}{6}\left(\frac{9K_{HM}+8G_{HM}}{K_{HM}+2G_{HM}}\right)}\right]^{-1}-\frac{G_{HM}}{6}\left(\frac{9K_{HM}+8G_{HM}}{K_{HM}+2G_{HM}}\right),
  \label{eq:kdry}
\end{equation}

where $\phi$ is a vector of porosities that range between 0.0 and 0.4. \ref{fig:theoryModel}
 shows the $K_{dry}$ moulus that corresponds to the modified HS lower bound or friable sand model.

\subsubsection{Cemented sand model}

This model was developed by \cite{ref:dvorkin2}, consists in finding the elastics moduli 
of the dry rock when there is no cement in the grain contacts. The cementation is product
of the diagenesis. The cement reduces porosity and increases the rock stiffness and permeability \citep{ref:avseth}. The effective moduli for the dry rock are given by: 


\begin{equation}
K_{dry}=\frac{n(1-\phi_{C})M_{c}S{n}}{6},
  \label{eq:kdryc}
\end{equation}

\begin{equation}
G_{dry}=\frac{3K_{dry}}{5}+\frac{3n(1-\phi_{C})G_{c}S_{\tau}}{20},
  \label{eq:kdryc}
\end{equation}

where $G_{c}$ is the shear modulus of the cement, n is the coordination number. $M_{c}$ is the compressional modulus of the cement given by:
\begin{equation}
M_{c}=K_{c}+\frac{4G_{c}}{3},
  \label{eq:mc}
\end{equation}

where $K_{c}$ is the bulk modulus of te cement.

The terms $S_{n}$ and $S_{\tau}$ corresponds to:
\begin{equation}
S_{n}=A_{n}\alpha^2+B_{n}\alpha+C_{n},
  \label{eq:mc}
\end{equation}

\begin{equation}
S_{\tau}=A_{\tau}\alpha^2+B_{\tau}\alpha+C_{\tau},
  \label{eq:mc}
\end{equation}


\begin{equation}
A_{n}=-0.024153\Lambda_{n}^{-1.3646}
  \label{eq:mc}
\end{equation}

\begin{equation}
B_{n}=0.020405\Lambda_{n}^{-0.89008}
  \label{eq:mc}
\end{equation}

\begin{equation}
C_{n}=0.000024649\Lambda_{n}^{-1.9864}
  \label{eq:mc}
\end{equation}


\begin{equation}
A_{\tau}=-10^{2}(2.26\upsilon_{s}^2+2.07\upsilon_{s}+2.3)\Lambda_{\tau}^{0.079\upsilon_{s}^2+0.1754\upsilon_{s}-1.342},
  \label{eq:mc}
\end{equation}

\begin{equation}
B_{\tau}=(0.0573\upsilon_{s}^2+0.0937\upsilon_{s}+0.202)\Lambda_{\tau}^{0.0274\upsilon_{s}^2+0.0529\upsilon_{s}-0.8765},
  \label{eq:mc}
\end{equation}

\begin{equation}
C_{\tau}=10^{-4}(9.654\upsilon_{s}^2+4.945\upsilon_{s}+3.1)\Lambda_{\tau}^{0.01867\upsilon_{s}^2+0.4011\upsilon_{s}-1.8186},
  \label{eq:mc}
\end{equation}


\begin{equation}
\Lambda_{n}=\frac{2G_{c}(1-\upsilon_{s})(1-\upsilon_{c})}{G\pi(1-2\upsilon_{c})}
\label{eq:mc}
\end{equation}

\begin{equation}
\Lambda_{\tau}=\frac{G_{c}}{G\pi}
\label{eq:lambda}
\end{equation}

the terms $\upsilon_{s}$ and $\upsilon_{c}$ are the poisson ratio of the rock dry mineral frame and the cement, 
respectively. $\alpha$ is the amount of contact cement for cemented layer around the grain:


\begin{equation}
\alpha=\left[\frac{\frac{2}{3}(\phi_{C}-\phi)}{1-\phi_{C}}\right]^{0.5}
\end{equation}

\ref{fig:theoryModel} shows the $K_{dry}$ modulus that corresponds to the contact cement model.
The reader is referred to \cite{ref:dvorkin2} for a complete review of the cemented model.

Once the contact cement model is built constant cementation lines are calculated using \cite{ref:avseth2} approach.
\cite{ref:avseth2} suggested that all sands have the same amount cement at different porosities. The porosity reduction
 is related to sorting and cementation. The effective moduli are given by: 

\begin{equation}
K_{dry}=\left[\frac{\frac{\phi}{\phi_{b}}}{K_{b}+\frac{4}{3}G_{b}}+\frac{1-\frac{\phi}{\phi_{b}}}{K+\frac{4}{3}G_{b}}\right]^{-1}-\frac{4}{3}G_{b},
  \label{eq:kdry}
\end{equation}

\begin{equation}
G_{dry}=\left[\frac{\frac{\phi}{\phi_{b}}}{G_{b}+\frac{G_{b}}{6}\left(\frac{9K_{b}+8G_{b}}{K_{b}+2G_{b}}\right)}+\frac{1-\frac{\phi}{\phi_{b}}}{G+\frac{G_{b}}{6}\left(\frac{9K_{b}+8G_{b}}{K_{b}+2G_{b}}\right)}\right]^{-1}-\frac{G_{b}}{6}\left(\frac{9K_{b}+8G_{b}}{K_{b}+2G_{b}}\right),
  \label{eq:kdry}
\end{equation}


where $K_{b}$, $G_{b}$ and $\phi_{b}$ are the bulk, shear modulus and porosity calculated in the contact cement model.
\ref{fig:theoryModel} shows the $K_{dry}$ constant cementation lines.
 
\csmlongfigure[h!]{theoryModel}{./chapters/Chapter3_Rock_physics/Figures/theoryModel}{5in}{
Uncosolidated, contact cement and constant cement models. The black dot indicates the end member
of the unconsolidated model corresponding to $K_{HM}$ from Hertz Mindlin theory. The models are calculated
for 85\% of Quartz and 15\% of clay. The cement corresponds to quartz}


\subsection{Model calibration}

Wells 140-1 and 159-2, XRD, thin sections, core and facies data information were used to calibrate the model.
Bulk dry modulus was calculated in Chapter \ref{sec:drybulk}, shear dry modulus is assumed  to be
equal to the saturated shear bulk modulus. Elastic dry moduli porosity trends are evaluated
in \ref{fig:model_calibration}. One can see that shaly sandstones are clearly discriminated from
the cleanest sandstones. The reservoir sandstones have different trends, for instance one can 
observe the cleaner sandstones are stiffer. This is because they composition is mainly quartz 
and also may be due to cemented grains. 

\csmlongfigure[h!]{model_calibration}{./chapters/Chapter3_Rock_physics/Figures/model_calibration}{7in}{
Dry elastic moduli from wells 140-1 and 159-2, the colorbar indicates the volume of shale. The unconsolidated,
 contact cement and constant cement models are indicated by the blue lines. The models are calculated
for 85\% of Quartz and 15\% of clay. The cement corresponds to quartz}


The same graphic using the facies
as the color code  is shown in \ref{fig:model_facies}. Now, one can observe that is easier to analyze elastic
moduli porosity trends. These trends can be related to the facies. From this figure, 
three groups that follow similar trends are identified: Paluxy-Tuscaloosa washover, 
Tuscaloosa fluvial and Tuscaloosa beach facies. The composition of each group
and the modeling parameters are shown in \ref{tab:modeling}.



\csmlongfigure{model_facies}{./chapters/Chapter3_Rock_physics/Figures/model_facies}{6in}{
Dry elastic moduli from wells 140-1 and 159-2, the colors indicate the facies interpreted by
\cite{ref:doug}. The unconsolidated,
 contact cement and constant cement models are indicated by the blue lines. The models are calculated
for 85\% of Quartz and 15\% of clay. The cement corresponds to quartz}

\begin{table}
\caption{ Modeling parameters \label{tab:modeling}}
\begin{center}
\begin{tabular}{ | l | l | l|  l | l | l|}
\hline
  Facies                               &  n    & Quartz & Clay  & Siderite & P$eff$ (Mpa)  \\
\hline
Tusc Washover/Transitional and Paluxy  & 10.36 &   0.85 &  0.15 &     0.0  &  12.5  \\
\hline
Tusc Fluvial                           & 10.36 &   0.7  &  0.2  &     0.1  &  12.5  \\
\hline
Tusc Beach/Barrier                     & 10.36 &   0.97  &  0.03 &     0.0  &  12.5  \\
\hline
\end{tabular}
\end{center}
\end{table}

\subsubsection{Paluxy-Tuscaloosa washover facies}

Paluxy and Tuscaloosa washover facies were grouped, both of this formations
have similar compositions around 85\% of quartz and 15\% of clay. \ref{fig:facies_modelingP}
shows the shear bulk modulus for these facies. One can note that the model falls mainly 
in the red line that indicates unconsolidated sand model. \ref{fig:thinTW} shows the thin section of 
a typical Tuscaloosa washover facies for well 159-2. Minor compaction and no cementation
is observed. Therefore, the unconsolidated sand model is the appropiated model for
the study of these facies.  


\csmlongfigure[h!]{facies_modelingP}{./chapters/Chapter3_Rock_physics/Figures/facies_modelingP}{5in}{
Shear modulus for the Paluxy and Tuscaloosa washover facies. The red line indicates
the unconsolidated sand model. Blue lines indicate the contact and constant cement models.
The models are calculated for 85\% of Quartz and 15\% of clay. The cement corresponds to quartz}



\csmlongfigure[h!]{thinTW}{./chapters/Chapter3_Rock_physics/Figures/thinTW}{5in}{
Typical thin section of the Tuscaloosa washover facies for core 159-2. Note there is minor
compaction and no cementation}

\subsubsection{Tuscaloosa fluvial facies}

Tuscaloosa fluvial facies is the one that have more clay content from the reservoir
sandstones. The average composition is 70\% of quartz, 20\% of clay and 10\% of siderite.
\ref{fig:facies_modelingTF} shows the shear modulus for this facies. The model fits the
unconsolidated sand model. The thin sections shown in \ref{fig:thinTF} reveals abundant clay and
siderite, minor compaction and no cementation.


\csmlongfigure[h!]{facies_modelingTF}{./chapters/Chapter3_Rock_physics/Figures/facies_modelingTF}{5in}{
Shear modulus for the Tuscaloosa fluvial facies. The red line indicates
the unconsolidated sand model. Blue lines indicate the contact and constant cement models.
The models are calculated for 70\% of quartz, 20\% of clay and 10\% of siderite. 
The cement corresponds to quartz}

\csmlongfigure[h!]{thinTF}{./chapters/Chapter3_Rock_physics/Figures/thinTF}{5in}{
Typical thin section of the Tuscaloosa fluvial facies for core 159-2. Note there is
abundant clay and siderite}

\subsubsection{Tuscaloosa beach facies}

Tuscaloosa beach facies is the cleanest and stiffer of the reservoir
sandstones. The average composition is 97\% of quartz, and 3\% of clay.
\ref{fig:facies_modelingTB} shows the shear modulus for this facies. The model fits the
red line whic corresponds to quartz cementation of 3\%. The thin sections shown in \ref{fig:thinTB} reveals that there is contact between the quartz grain with possible patchy cement. Almost no clay
is present.

The contact cement and constant cement models do no take into account 
stress sensitivity. However, there is an important stress effect 
in the rocks that have patchy cementation where some grain are cemented and some are lose. 
\citep{ref:avsethcement} suggested a weight function that measures the degree of rock 
unconsolidation that allow to quantify the stress effect in cemented rocks:
 
\begin{equation}
W=\frac{K_{dry}-K_{soft}(P_{0})}{K_{stiff}-K_{soft}(P_{0})},
  \label{eq:mc}
\end{equation}

\begin{equation}
K_{dry}(P_{eff})=(1-W)K_{soft}(P_{eff})+WK_{stiff},
  \label{eq:mc}
\end{equation}

where $K_{dry}$ is the observed modulus that corresponds to 3\% of quartz cement, $K_{soft}$ is
the unconsolidated model for a given pressure $P_{0}$ and $K_{stiff}$ is modulus value when
the rock is insentive to stress that corresponds to the $K$ for rocks that have 10\% of cement. These equations
will be applied in the dynamic modeling when pressure sensitivity is measure for each facies.




\csmlongfigure[h!]{thinTB}{./chapters/Chapter3_Rock_physics/Figures//facies_modelingTB}{5in}{
Shear modulus for the Tuscaloosa beach facies. The red line indicates
3\% of quartz cement. Blue lines indicate the unconsolidated, contact and constant cement models.
The models are calculated for 97\% of quartz, and 3\% of clay.}

\csmlongfigure[h!]{facies_modelingTB}{./chapters/Chapter3_Rock_physics/Figures/thinTB}{5in}{
Typical thin section of the Tuscaloosa beach facies for core 159-2. Note there is
abundant quartz, almost no clay and possible patchy cementation.}

\clearpage
\newpage
\subsection{Static Rock Physics template}

Fluid substitution was applied using Gasmmann's equation \ref{eq:gasmmann} to calculate
the saturated moduli of rocks with 100\% brine, 100\% oil and 100\% $CO_{2}$. The fluid properties
and the elastic moduli used were shown in \label{tab:fluid} and \label{tab:flag} in the previous chapter.

$V_{p}/V_{s}$ ratio and acoustic impedance of the saturated rocks were calculated in 
order to built the static rock physics template (SRPT). 
The SRPTs are widely use in the industry for reservoir quantitative interpretation 
because allow to evaluate well log data and seismic inversion outputs in function
of elastic parameters. \ref{fig:static} shows the SRPT for each modeled facies group. The blue, 
green and red line indicate rocks saturated with 100\% brine, 100\%oil and 100\%$CO_{2}$, respectively. 
Hydrocarbon presence decrease considerable the $V_{p}/V_{s}$ ratio, especially the $CO_{2}$ due to its lower bulk
modulus and density.  
One can note that each
facies group have different $V_{p}/V_{s}$ ratio and acoustic impedance trends. Lower $V_{p}/V_{s}$ ratio
are observed for the Tuscaloosa beach facies and increase in the Paluxy-Tuscaloosa washover and Tuscaloosa
fluvial facies. Therefore, $V_{p}/V_{s}$ ratio is powerful lithology and fluid discriminator in Delhi Field. 

\csmlongfigure[h!]{static}{./chapters/Chapter3_Rock_physics/Figures/static}{7in}{Static rock physics template
for the Paluxy-Tuscaloosa washover facies, Tuscaloosa fluvial facies and Tuscaloosa beach facies.  The blue, 
green and red line indicate rocks saturated with 100\% brine, 100\%oil and 100\%$CO_{2}$, respectively. 
}

\subsection{Dynamic Rock Physics template}

Dynamic modeling has the objective of measuring changes in the reservoir
properties. In Delhi Field, I assume the changes in the reservoir are given due
to pressure and $CO_{2}$ saturation, compaction or dilation are not taken into account
in the modeling (find a reference or explain why I think compaction or dilation
will not affect the elastic properties as much as pressure and saturation changes). 

The scope of the dynamic rock physics modeling for Delhi field consist on
quantifying the relative changes between monitors 1 and 2, both of them shot
after $CO_{2}$ injection program started. Relative changes in the elastic properties
are given by: 

\begin{equation}
P_{diff}= 100*[P_{mon2}-P_{mon1}]/P_{mon1},
  \label{eq:changes}
\end{equation}

where $P_{diff}$ is any elastic property, $P_{mon2}$ and $P_{mon1}$ are the elastic properties
for monitor 2 and 1, respectively. \ref{fig:co2sat} shows how the relative changes in $V_p/V_s$ ratio,
 compressional velocity, shear velocity, acoustic impedance an shear impedance 
when $CO_2$ replaces 65\% brine and 35\% of oil for a constant pore pressure of
10 MPa. One can note, that few amount of $CO_{2}$ changes dramatically the compressional 
velocity, acoustic impedance and $V_{p}/V_s$ ratio but gradual when increasing the saturation.
 Shear velocity and shear impedance are almost insensitive to fluids.
Also, we can observe the non-linear behavior of
$CO_2$ which make it difficult to quantify but easy to identify in the seismic. Therefore, the purpose
of the study is to detect $CO_2$ and not quantifying it.
  

\csmlongfigure[h!]{co2sat}{./chapters/Chapter3_Rock_physics/Figures/co2sat}{5in}{}

Pressure changes are analyzed using a pore pressure reference of 13.0 MPa for a rock
65\%brine and 35\% oil saturated. \ref{fig:pressure} shows the relative changes 
in $V_p/V_s$, acoustic and shear impedance due to pressure changes. One can observe 
pore pressure changes are almost linear when pore pressure increases. However, below
the bubble point pressure gas will come out of solution and then two fluid phases will
be in the pore space. This aspect represents a limitation when predicting pressures. 
Also, one can observe that Vp/Vs ratio curves behave in the opposite direction 
from the rest of the properties. Therefore crossploting vp/vs ratio with 
any of the properties will allows to visualize better pore pressure changes. 

The analysis of pure pressure changes or pure saturation change does not represents
the reality of the reservoir, usually both changes occur at the same time. Figure XX
shows the effects in Vp/Vs ratio and acoustic impedance

 
\csmlongfigure[h!]{pressure}{./chapters/Chapter3_Rock_physics/Figures/pressure}{5in}{}

Then I built a dynamic rock physics template (DRPT) where I modeled the changes in the parameter
to mimic the changs in the reservoir due to co2 injection. I modeled both scenarios
at the same time, saturation and pressure change. Then I recognized that four quadrants
indicates mainly pore pressure increase, decrease and co2 and brine.

