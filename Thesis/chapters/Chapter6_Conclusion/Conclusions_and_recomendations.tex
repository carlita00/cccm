\chapter{Conclusions and Recommendations}

\subsection{Conclusions}

\begin{itemize}

\item The P-P and P-S post-stack joint inversion increased the resolution on the interpretation of the upper zone of the reservoir especially. The shear impedance and the inverted density improved delimitation of the sandstone bodies particularly in the lower zone of the reservoir.

\item In the lower reservoir unit, the inverted density and shear impedance attribute has a higher definition of the reservoir heterogeneities and properties than that derived from the P-P amplitude attribute map.

\item The use of P and S impedance and density volumes enable better reservoir heterogeneities delineation especially when quantified with seismic attributes. 

\end{itemize} 

\subsection{Recommendations}

\begin{itemize}

\item Continue with the reservoir characterization study doing a 3-D interpretation of all sandstone bodies using the results from the P-P and P-S post-stack inversion generated in this research.

\item Perform an elastic inversion using the P-P partial stacks to generate a new acoustic and shear impedance volumes and compare these results with the P-P and P-S post-stack results.

\item Perform an P-P and P-S elastic joint inversion with the available partial stacks.

 

\end{itemize}

