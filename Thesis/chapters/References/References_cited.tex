%\chapter{References cited}
\begin{selected-bibliography}

\begin{itemize}

\item Bruhn, C.L.M., (1993). High resolution stratigraphy, reservoir geometry, and facies characterization of cretaceous and tertiary turbidities from Brazilian passive margin basins: Ph.D. Thesis, McMaster University, Hamilton, Ontario, Canada, 299p.


\item Chapin, M. A., G. M. Tiller, and M. J. Mahaffie, (1996). 3-D architecture modeling using high-resolution seismic data and sparse well control; example from the Mars field, Mississippi Canyon area, Gulf of Mexico, in P. Weimer and T. L. Davis, eds., Application of 3-D seismic data to exploration and production: AAPG Studies in Geology 42, p. 123�132.

\item Guardado, L. R., A. R. Spadini, J. S. L. Brand�o, and M. R. Mello, (2000). Petroleum system of the Campos Basin, in M. R. Mello and B. J. Katz, eds. , Petroleum systems of South Atlantic margins: AAPG Memoir 73, p. 317�324.


\item Guardado, L. R., L. A. P. Gamboa, and C. F. Lucchesi, (1989). Petroleum geology of the Campos Basin, Brazil: a model for producing Atlantic type basins, in J. D. Edwards and P. A. Santogrossi, eds. , Divergent/passive margin basins: AAPG Memoir 48, p. 3�80.

\item Guardado, L. R., W. E. Peres, and C. F. S. Cruz, (1986), Depositional model and seismic expression of turbidites in Campos basin, offshore Brazil: American Association of Petroleum Geologists Bulletin, v. 70, p. 597.


\item Hampson, D., J. Schuelke, and J. Quirein (2001). Use of multiattribute transforms to predict log properties from seismic data. Geophysics 66, 220-236.


\item Hampson and Russell (2008a). ISMAP Guide. Hampson and Russell.


\item Hampson and Russell (2008b). PROMC Guide. Hampson and Russell.


\item Lima, K.T.P., V.M. Reis, and S.R. Malagutti, (2010). Petrobras internal report confidential. 

\item Mello, M.R.,(1988). Geochemical and molecular studies of the depositional environments of source-rocks and their derived oils from the Brazilian marginal basins: Ph.D. dissertation, Bristol University, Bristol, U.K., 240p.

\item Mello, M. R., W. U. Mohriak, E. A. M. Koutsoukos, and G. Bacoccoli, (1994) Selected petroleum systems in Brazil, in L. B. Magoon and W. G. Dow, eds. , The petroleum system�from source to trap: AAPG Memoir 60, p. 499�512.

\item Mutti, E., (1992). The basic characteristics of turbidites and contourites, in Turbidite Sandstone, Istituto di Geologia Universita di Parma.

\item Ojeda, H.A.O., (1982). Structural Framework, Stratigraphy, and Evolution of Brazilian Marginal Basins, AAPG, v.66, n.6, p.732-749.

\item PGS Geophysical Acquisition Report, 2005. 3D 4C OBS Survey PGS Brasil, Petrobras confidential.

\item Pickering, K.T., R.N. Hiscott and F.J. Hein (1989). Submarine fans, in Deep Marine Environments: Clastic Sedimentation and Tectonics: Unwin Hyman, pg.169-178. 

\item Ponte, F. C., Fonseca, J. R., and Carozzi, A. V., (1980). Petroleum habitats in the Mezosoic-Cenozoic of the continental margin of Brazil: in Fact and Principles of World Petroleum Occurrence. A.D. Miall (Ed.), Fad. Soc. Petrol. Geol., Mem. 6, p. 857-886.


\item Posamentier, H. W., and Erskine, R.D., (1991). Seismic expression and recognition criteria of ancient submarine fans in P.Weimer and M.H.Link, ed., Seismic Facies and Sedimentary Process of Submarine Fans and Turbidite System.

\item Stow, D.A.V. ( 1992). Processes: overview and commentary, in Deep-Water Turbidite Systems, Volume 3 of the International Association of Sedimentologists.

\item Walker, R.G. (1978). Deep-water sandstone facies and ancient submarine fans: model for exploration for stratigraphic traps, AAPG Bulletin, v.62, n.6, p. 932-966.

\item Weimer, P., A. H. Bouma, and B. F. Perkins, eds., (1994a). Submarine fans and turbidite systems: Gulf Coast Section SEPM 15th Annual Research Conference, 440 p.

\item WesternGeco  Data �Processing�Report, �2010. Data processing report for Petrobras, confidential.

\item Winter, R.W., R.J. Jahnert and A. B. Franca, (2007). Campos Basin stratigraphic chart in Boletim Geociencias Petrobras, Rio de Janeiro, v. 15, n. 2, p. 511-529. 

\end{itemize}

\end{selected-bibliography}











