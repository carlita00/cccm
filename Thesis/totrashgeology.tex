parishes of northeastern Louisiana. The field is approximately 30 miles east of Monroe,
Louisiana and 40 miles west of the Louisiana-Mississippi Border



Delhi Field is located in the northeastern Louisiana (Figure 1.1). It is surrounded
to the west by the North Louisiana Salt Basin (NLSB), to the north by the Monroe uplift
and to the southeast by the Mississippi Interior Salt Basin (MISB). The field has been
undergoing CO2 injection since 2009 and the source of the CO2 is the Jackson Dome
which is located to the east of the field (Figure 1.2).

It
is also known as the “West Delhi” mainly because of its geographic location in regards to
the “Delhi on the east” and “Big Creek” to the west (Bloomer Jr., 1946). West Delhi is
about 15 miles long and 2 to 2.5 miles wide. Its original oil in place (OOIP) is about 357
MMBB. The primary recovery from 1944 to 1953 was about 49 MMBbl while secondary
recovery from 1953 to 1980 was about 143 MMBbl. The cumulative production was 192
MMBbl with a recovery factor of 54% of the OOIP. The peak production was 17,500
BOPD in 1970. The production stopped around 2005 (Figure 1.3). However, there was
still a great amount of residual oil left in the field. A tertiary recovery method was needed
to recover that oil.
Denbury Resources Inc. acquired


Delhi Field is located in northeastern Louisiana, 30 miles east of Monroe, Louisiana
 (Figure 1.1). The field was discovered by C.H. Murphy and Sun Oil Company in 1944.
 Delhi Field is classified as giant with an estimated EUR of 357 million barrels, 
covering an area approximately 15 miles long by 2.5 miles wide (Powell 1972). The 
primary reservoir zone is termed the Holt-Bryant zone and consists of Early Cretaceous
 Paluxy formation sandstones unconformably overlain by Late Cretaceous Tuscaloosa 
formation sandstones. The Tuscaloosa sandstones are amalgamated and exhibit high spatial
 variability. A regional stratigraphic column and type log for Delhi Field are shown in
 Figure 1.2. The primary trapping mechanisms include a depositional pinchout between Paluxy
 and Tuscaloosa sediments and an upper erosional angular unconformity against the Monroe Gas
 Rock, which provides a seal for the Holt-Bryant reservoir zone (Silvis 2011).


Delhi Field is located in northeastern Louisiana. Figure 1.1 It was discovered in
December 1944, by C.H. Murphy, Jr. and Sun Oil Company. The main reservoir is the
Holt-Bryant zone of the Paluxy Formations. Reservoirs consists of Upper Cretaceous
(Tuscaloosa) and Lower Cretaceous (Paluxy) sandstones. Initial production was 505
bbl of 41.7 API gravity oil with tubing pressure 625 psig, and the oil-gas ratio was 530
std. c.f. per bbl at depth 3280 to 3290 ft. Delhi Field is 11 miles long and 1.5 miles
wide. It is divided into three sub-elds: West-Delhi, Big Creek and East Delhi. The
Big Creeks production is from Marine Tuscaloosa sandstone, while producing sands
in East and West Delhi are from both Tuscaloosa and Paluxy Formations. West Delhi
2


Delhi Field is a case of study of the Reservoir Characterization Project (RCP)
since 2009, it is an excellent example for an enhanced oil recovery (EOR) 
active monitoring field with 4D multicomponent seismic technologies.The EOR technique 
consists on a CO$_2$ continuos flooding started in 2009 by
the operating company, Denbury Resources Inc. 


Delhi Field is located on the southern periphery of the Monroe uplift. The eld
is located in a structural sag between, two highs. The Rayville high is ve miles
east of Delhi Field, and the Epps high is sixteen miles southeast of the eld. The
Monroe uplift is a complexly truncated dome; the structure is approximately eighteen
miles in diameter. The age of the Monroe uplift is post-Jurassic (Ewing, 2001). It is
located within the Mississippi Alluvial Valley and it has a thick mantle of Quaternary
alluvium approximately 250 ft in thickness. Another structural feature near Delhi
Field is the Jackson Dome. Figure 1.2 The importance of the Jackson dome for the
eld is that it is the source of the CO2 which is being injected at Delhi Field. Jackson
dome is a buried volcano, which was discovered in 1860 (Mancini and Puckett, 2000).
The volcanic activity from the Cretaceous period in this region has been documented
(Hunter and Davies, 1979). However, volcanic activity has not aected much of the
3
Figure 1.2: Regional map showing the location of Delhi Field, also nearby structural
features are shown (Mancini, 2008).
productivity of Delhi Field. Some of the wells that were dry holes were in contact
with a diatreme or dike, b




This field represents a complex
stratigraphy siliciclastic reservoir, with high porosities and permeabilities
where the challegenges consists on mapping the CO$_2$ in order to maximize
recovery.

Delhi is located in 

The reservoir  is formed by the Paluxy and Tuscaloosa formations. These
sandstones have been interpreted as fluvio deltaic and shallow marine 
respectively with thickness ranging from 30 to 85 feet (Silvis, 2011). 
Previous studies has shown poor lateral continuity in the Tuscaloosa (Silvis, 2011),
therefore the dynamic characterization of this complex 
reservoir is a challenge.

